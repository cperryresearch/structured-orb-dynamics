
% ======================================
% MAIN MANUSCRIPT STRUCTURE (CLEANED)
% ======================================

\documentclass[11pt]{article}
% ============================
% Shared preamble for all papers
% Structured Orb Dynamics — Unified Manuscript & Data Repository
% ============================

% Encoding & fonts
\usepackage[utf8]{inputenc}
\usepackage[T1]{fontenc}
\usepackage{lmodern}

% Page layout
\usepackage[margin=1in]{geometry}
\usepackage{setspace}
\onehalfspacing

% Math packages
\usepackage{amsmath, amssymb, amsthm, mathtools}
\usepackage{bm}

% Figures & graphics
\usepackage{graphicx}
\usepackage{grffile}
\usepackage{subcaption}

% Tables
\usepackage{booktabs}
\usepackage{multirow}
\usepackage{siunitx}
\sisetup{detect-all}

% Lists
\usepackage{enumitem}

% Colors
\usepackage{xcolor}
\definecolor{linkblue}{HTML}{0B4F6C}

% Hyperlinks
\usepackage{hyperref}
\hypersetup{
    colorlinks   = true,
    linkcolor    = linkblue,
    citecolor    = linkblue,
    urlcolor     = linkblue,
    pdftitle     = {Structured Orb Dynamics},
    pdfauthor    = {Cassandra Perry},
    pdfproducer  = {LaTeX with Orb Motion Classifier},
}

% Clever references
\usepackage{cleveref}

% Theorem-like environments
\newtheorem{definition}{Definition}
\newtheorem{theorem}{Theorem}
\newtheorem{proposition}{Proposition}
\newtheorem{lemma}{Lemma}

% Custom macros
\newcommand{\R}{\mathbb{R}}
\newcommand{\curv}{\kappa}
\newcommand{\jerk}{j}

% Motion-state notation
\newcommand{\state}{\mathcal{S}}
\newcommand{\orbstate}{\ensuremath{\mathrm{orb}}}
\newcommand{\hoverstate}{\ensuremath{\mathrm{hover}}}
\newcommand{\straightstate}{\ensuremath{\mathrm{straight}}}
\newcommand{\turnstate}{\ensuremath{\mathrm{turn}}}

% TODO comments
\newcommand{\todo}[1]{\textcolor{red}{[TODO: #1]}}

% Bibliography
\usepackage[numbers,sort&compress]{natbib}

% Optional visual polish
\usepackage{microtype}

\begin{document}

\title{Structured Orb Dynamics:\\ Unified Manuscript and Data Repository}
\author{Cassandra Perry}
\date{\today}

\maketitle

\begin{abstract}
Infrared videos of airborne objects are often hard to analyze in a formal sense. 
Most of them offer almost no information about range, calibration, or platform motion, 
so many of the usual tools for interpreting motion are simply unavailable. 
Still, the image plane carries enough geometric structure to say something about how an 
object’s apparent motion changes over time, as long as the analysis stays anchored to what 
the footage actually shows.

Structured Orb Dynamics is a geometry–first framework built with that constraint in mind. 
It reconstructs a stabilized trajectory, evaluates quantities such as curvature, short-term 
directional changes, and apparent speed, and then groups the motion into a small set of 
geometric states. The aim is not to extract forces or suggest what the object might be. 
It is to provide a clear and repeatable way to describe the motion when physical 
information is missing.

The PR--018 infrared recording serves as an example of how this works in practice. 
Because the target remains visible and roughly stabilized, the method highlights long 
straight stretches, a sustained turning interval, and several brief low-velocity moments 
where the geometry becomes harder to interpret. These findings do not identify the object, 
but they do outline what the video reliably contains and where its interpretive limits begin.
\end{abstract}

\section{Methods}
\label{sec:methods}

\subsection{Method Scope and Positioning}
\label{subsec:methods_scope}

Structured Orb Dynamics (SOD) is applied after a trajectory has already been reconstructed. In
practice, the method is used to examine how the apparent motion of a tracked point changes over
time. SOD does not perform tracking itself. Instead, it evaluates whether the geometry of a
trajectory supports stable interpretations such as straight motion, sustained turning, or
intervals where apparent displacement becomes very small.

The framework is agnostic to how trajectories are obtained. In this work, it is applied to
stabilized image-plane tracks derived from infrared video, but the same analysis can be performed
on trajectories produced by learned trackers, optical-flow methods, or manual annotation. What
matters is not visual continuity, but whether geometric behavior persists long enough to support
interpretation. Persistence is enforced using explicit minimum-run requirements described in
Part~III. When this condition is not met, segments are left unclassified, and trajectories may
fragment or collapse as the geometry becomes unreliable.

All outputs produced by SOD are descriptive. State labels summarize patterns in image-plane motion
only; they are not causal claims and are not intended to represent physical forces, control
mechanisms, object identity, or intent.

\subsection{Inputs and Assumptions}
\label{subsec:methods_inputs}

The primary input to the method is a two-dimensional trajectory expressed in image-plane
coordinates and sampled at discrete time steps. Each position is assumed to correspond to the same
tracked point or object across frames; the method does not attempt to resolve identity beyond this
geometric consistency.

The analysis operates entirely in the image plane. No camera calibration, range information, or
external metadata is required. No assumptions are made regarding mass, propulsion, aerodynamic
forces, or sensor-specific dynamics. All quantities used in classification are derived directly
from geometric relationships within the observed trajectory and are interpreted only in that
context.

\subsection{Procedural Overview}
\label{subsec:methods_overview}

The analysis proceeds in three stages. First, a trajectory is reconstructed and smoothed to reduce
high-frequency tracking noise (Part~I). From this stabilized path, geometric quantities such as
curvature and apparent velocity are computed (Part~II). Motion states are then assigned using
explicit geometric criteria and temporal persistence requirements, yielding a discrete sequence of
geometric regimes (Part~III).

The following sections apply this procedure to empirical data used in this study and examine its
behavior under both representative examples and control conditions.

\subsection{Audit Boundary and Withholding Demonstration}
\label{subsec:methods_audit_boundary}

An essential feature of Structured Orb Dynamics is its ability to withhold classification when the observed geometry does not support a stable or persistent motion-state assignment. Rather than forcing labels onto short, noisy, or weakly constrained segments, the framework treats \mbox{non-classification} as a valid and informative outcome. This behavior functions as an internal audit boundary: it marks regions of the trajectory where the available image-plane information is insufficient to sustain geometric interpretation under the persistence criteria defined in Part~III. The purpose of this demonstration is not to introduce a new method but to make this withholding behavior explicit and inspectable.

\begin{figure}[h]
    \centering
    \includegraphics[width=\linewidth]{figures/pcmsat_withhold_visual.png}
    \caption{Withholding outcome under the Structured Orb Dynamics decision rules. The \mbox{reconstructed} trajectory and associated geometric signals are shown for a segment in which curvature, apparent speed, or temporal persistence do not meet the minimum requirements for stable \mbox{motion-state} assignment. No motion label is produced for this interval. This figure illustrates the intentional audit boundary of the method: when geometric support is insufficient, classification is withheld rather than inferred.}
    \label{fig:pcmsat_withhold}
\end{figure}

The underlying audit artifact associated with this demonstration is archived separately in a public repository for transparency and reference, without introducing coupling or implementation dependencies within the present manuscript.

\section*{Contributions}

This work develops a unified, geometry–only framework for describing motion in infrared videos that lack physical calibration. The main contributions are:

\begin{enumerate}
    \item \textbf{A reproducible trajectory reconstruction pipeline.}  
    We outline a minimal and transparent procedure for extracting and smoothing image-plane trajectories from stabilized infrared recordings. The steps are intentionally simple so that they can be followed, adapted, or audited without relying on metadata that is not available in the footage.

    \item \textbf{A set of geometric quantities suited to limited information.}  
    The framework focuses on curvature, directional change, and apparent speed—signals that remain interpretable even when scale, range, and platform telemetry are unknown. These quantities form the basis for all later analysis and tie the method to what the footage directly supports.

    \item \textbf{A motion-state classifier and associated transition structure.}  
    Local segments of the trajectory are grouped into Straight, Turn, Hover, or Orb regimes using a modest and interpretable classifier. A simple transition logic summarizes how these regimes evolve over time. Both components are designed to reflect geometric behavior alone and avoid assumptions about underlying physical mechanisms.

    \item \textbf{A detailed application to the PR--018 dataset.}  
    The full Structured Orb Dynamics workflow is applied to a publicly released infrared recording to illustrate what can—and cannot—be inferred from geometric information alone. The results show long straight intervals, a sustained turning segment, and brief low-velocity moments where the geometry becomes harder to interpret. These observations demonstrate both the strengths and the limits of the framework under real observational conditions.
\end{enumerate}

Taken together, these components form a coherent system for describing motion when only the image plane is available. The goal is not to identify the object or explain its cause, but to document its observable structure in a consistent and repeatable way.

\tableofcontents
\clearpage

% ======================================
% PART I – INSTRUMENTATION & PIPELINE
% ======================================

\section*{Part I: Instrumentation and Reconstruction Pipeline}
\addcontentsline{toc}{section}{Part I: Instrumentation and Reconstruction Pipeline}
% ============================
% PART I — MODEL AND DATA
% ============================

\section{Orb Motion Classifier: Dynamical Motion-State Model}
\label{sec:orb-motion-classifier}

\section{Introduction}

Publicly released infrared recordings of unidentified aerial phenomena (UAP) provide a rare 
opportunity to study motion under conditions where physical metadata, calibration information, and 
platform telemetry are incomplete or unavailable. Although such datasets are limited, they still 
contain observable geometric structure that can be analyzed without relying on assumptions about 
the underlying mechanism or intent of the object.

The aim of this work is to develop a clear and reproducible framework for describing aerial motion 
using only the information that can be reliably extracted from image-plane data. The focus is not 
on classification in the traditional sense, nor on proposing physical explanations, but on building 
a structured way to document how an object's apparent motion evolves over time.

Our approach integrates three components:
\begin{enumerate}
    \item trajectory reconstruction from stabilized infrared footage,
    \item curvature-based geometric diagnostics of local motion,
    \item a discrete motion-state model that organizes observed behavior into simple patterns.
\end{enumerate}
Together, these elements form a unified framework that allows motion to be described in terms of 
observable kinematic structure rather than speculative interpretation.

The PR018 infrared video serves as the primary case study because it contains enough stabilization 
around the target to permit construction of an approximate image-plane trajectory. This makes it 
possible to estimate curvature, examine how curvature evolves, and evaluate which types of motion 
are compatible with the observed behavior.

To illustrate the limits of the method, a supplementary analysis is applied to the GIMBAL dataset. 
Because strong parallax and missing metadata prevent trajectory reconstruction, the GIMBAL analysis 
is necessarily qualitative and conservative. This contrast highlights the conditions under which the 
framework can and cannot be applied.

Throughout the manuscript, the emphasis is on transparency, reproducibility, and proportional 
inference. The framework is intentionally simple and geometric, designed to describe motion in a 
way that remains valid even when only partial information is available. No claims are made about 
what the object is, how it is propelled, or whether it corresponds to any particular physical model. 
The goal is to provide a consistent language for discussing observable motion in datasets where 
traditional physical inference is not possible.

By presenting the methodology alongside both a quantitative case (PR018) and a constrained, 
qualitative case (GIMBAL), the manuscript aims to show how careful, geometry-based analysis can help 
clarify what can—and cannot—be extracted from limited infrared recordings.

In this section we introduce the formal dynamical motion-state model that underpins the Orb Motion Classifier. We first define the reconstructed trajectory and associated kinematic quantities, then specify the discrete motion states, the per-state likelihood model, and the temporal smoothing structure.

\subsection{Video Products}
\label{subsec:video_products}

The PR-018 dataset consists of a stabilized infrared video recorded by a Department of Defense
platform and released as part of the UAP Task Force materials (2020). The video contains 
approximately 2{,}781 frames of mid-wave infrared imagery with an estimated frame rate of 
$\sim$30\,Hz. The exact sensor aperture, focal length, and field-of-view parameters have not been 
publicly released; however, the imagery exhibits the characteristic contrast profile of MWIR 
point-source tracking. The video appears to be cropped around the target, and no ancillary 
telemetry—such as range, platform altitude, or gimbal angles—has been provided.

Due to missing metadata, certain parameters (e.g., exact frame rate, zoom level, and sensor 
stabilization characteristics) are inferred indirectly from frame timing and visual inspection. 
Where uncertainties exist, we report conservative estimates and emphasize the assumptions used in 
trajectory reconstruction.

% (Continue any remaining Part I sections here, such as trajectory reconstruction, 
% curvature estimation details, classifier outputs, figures, etc.)

% ============================
% END OF PART I
% ============================


\clearpage

% ======================================
% PART II – THEORETICAL FRAMEWORK
% ======================================

\section*{Part II: Theoretical Model}
\addcontentsline{toc}{section}{Part II: Theoretical Framework}
\label{sec:part2_intro}
% ==============================
% PART II – THEORETICAL FRAMEWORK
% ==============================

Part~II develops the mathematical and conceptual foundations of the Orb Motion Classifier.
Where Part~I focused on constructing empirical trajectories, estimating curvature, and evaluating
state likelihoods, this part formalizes the geometric principles that make those operations
coherent. Here we introduce the theoretical structure that explains why the classifier behaves
as it does, how curvature and its derivatives encode dynamical information, and under what
conditions the resulting motion-state framework can generalize across heterogeneous datasets
with incomplete metadata.

The goal of this part is to unify curvature-based diagnostics, discrete motion-state modeling, and 
probabilistic classification under a single coherent analytical framework. This enables rigorous 
interpretation of state transitions, diagnostic consistency across datasets, and transparent 
assumptions for future extensions.

% --------------------------------
% UNIFIED THEORY FRAMEWORK SECTION
% --------------------------------

\section{Unified Theory Framework}
\label{sec:unified_framework}

This section introduces the theoretical architecture that unifies curvature-based motion analysis, 
discrete dynamical state modeling, and probabilistic classification into a single interpretable 
framework. The objective is to establish a mathematically consistent foundation for analyzing 
unknown aerial trajectories and comparing their dynamical signatures across heterogeneous datasets.

% --------------------------------
% 2.1 Overview of Motion-State Geometry
% --------------------------------

\subsection{Overview of Motion-State Geometry}
\label{subsec:geometry_overview}

We define four fundamental motion states characterized by geometric invariants:

\begin{itemize}
    \item \textbf{Straight State ($S$)}: low curvature, stable direction of motion.
    \item \textbf{Turn State ($T$)}: sustained nonzero curvature with consistent turning direction.
    \item \textbf{Hover State ($H$)}: minimal displacement and low-speed regime.
    \item \textbf{Orb State ($O$)}: smoothly varying curvature with continuous higher derivatives.
\end{itemize}

Each state corresponds to a qualitatively distinct dynamical behavior and is identified through 
features derived from velocity, acceleration, and curvature signals.

These geometric distinctions form the foundation of the unified framework by mapping observable 
trajectory features to interpretable dynamical categories.

% --------------------------------
% 2.2 Curvature as a State Variable
% --------------------------------

\subsection{Curvature as a State Variable}
\label{subsec:curvature_state}

Curvature plays a central role in the unified motion-state framework because it provides a 
coordinate-invariant measure of how a trajectory departs from straight-line motion. Unlike 
velocity or acceleration—both of which depend on external reference frames—curvature is an 
intrinsic geometric property of the path itself. It therefore offers a stable discriminator of 
motion regimes, even when sensor metadata or absolute position information is incomplete.

Let $x(t) = (x_1(t), x_2(t))$ denote the reconstructed 2D trajectory in image-plane coordinates. 
The instantaneous curvature is defined as:
\begin{equation}
    \kappa(t) = 
    \frac{|x'(t) y''(t) - y'(t) x''(t)|}
    {\big( x'(t)^2 + y'(t)^2 \big)^{3/2}},
    \label{eq:curvature}
\end{equation}
with the standard extension to 3D trajectories as needed. Curvature evaluates how sharply the 
trajectory bends at each moment; $\kappa(t) \approx 0$ corresponds to straight-line motion, while 
larger values indicate progressively tighter turning.

\paragraph{Curvature Rate and Jerk Minimization.}
The temporal derivative $\dot{\kappa}(t)$ captures how curvature evolves over time. In many 
natural and engineered systems, high-frequency fluctuations in $\dot{\kappa}(t)$ are suppressed 
due to energetic or physical constraints, producing trajectories that minimize jerk. This yields:
\begin{itemize}
    \item slowly varying curvature in straight and turning motion,
    \item noisy or unstable curvature estimates during hover,
    \item smooth, bounded curvature evolution in the orb regime.
\end{itemize}

\paragraph{Curvature Under Measurement Noise.}
Because curvature depends on first and second derivatives, it is sensitive to noise. To mitigate 
this, curvature estimation proceeds through:
\begin{enumerate}
    \item temporal smoothing of the trajectory,
    \item finite-difference derivative estimation,
    \item normalization to account for varying speed,
    \item thresholding when velocity approaches zero.
\end{enumerate}

These steps preserve the qualitative structure of the curvature trace while suppressing 
frame-to-frame irregularities. Even under moderate noise, curvature retains the essential 
patterns needed to discriminate between straight, turning, hovering, and orb-like motion.

\paragraph{Why Curvature is Foundational.}
Curvature is the logical basis for the unified model because it offers:
\begin{itemize}
    \item \textbf{Geometric Invariance:} independence from coordinate frames.
    \item \textbf{State Separation:} each motion regime produces a distinct curvature profile.
    \item \textbf{Probabilistic Compatibility:} $\kappa(t)$ and $\dot{\kappa}(t)$ integrate 
          naturally into likelihood models.
\end{itemize}

Curvature therefore provides an interpretable, noise-tolerant bridge between observable kinematic 
features and the dynamical state model described next.

% --------------------------------
% 2.3 Discrete Dynamical State Model
% --------------------------------

\subsection{Discrete Dynamical State Model}
\label{subsec:state_model}

While curvature provides an instantaneous geometric description of motion, many behaviors of 
interest unfold over extended time intervals. To capture these temporal dependencies, we model 
the evolving motion regime as a discrete state process
\[
    S_t \in \mathcal{S} = \{ S,\, T,\, H,\, O \},
\]
where $S$, $T$, $H$, and $O$ denote the straight, turn, hover, and orb states, respectively. The 
state variable $S_t$ summarizes the dominant dynamical behavior at time $t$, integrating local 
geometric cues with temporal smoothing that reflects physical continuity.

\paragraph{State-Transition Structure.}
We assume that state evolution follows a first-order Markov process:
\[
    \mathbb{P}(S_{t} = s' \mid S_{t-1} = s,\, x_{1:t}) 
    = \mathbb{P}(S_{t} = s' \mid S_{t-1} = s),
\]
where $x_{1:t}$ denotes all observations up to time $t$. Although the classifier incorporates 
observational likelihoods, the transition structure itself depends only on the preceding state. 
This assumption reflects the notion that sudden, high-frequency transitions between strongly 
different dynamical regimes are physically improbable.

The transition matrix encodes:
\begin{itemize}
    \item high self-transition probabilities (temporal persistence),
    \item moderate transitions among compatible states (e.g., $S \leftrightarrow T$),
    \item suppressed transitions between incompatible states (e.g., $H \rightarrow S$ without 
          acceleration),
    \item rare transitions into or out of the orb state, reflecting its specialized geometry.
\end{itemize}

\paragraph{State-Conditional Likelihood Models.}
Each state generates characteristic patterns in curvature, velocity, and acceleration. We model 
the likelihood of observing the geometric features $f_t$ at time $t$ under each state as
\[
    \mathcal{L}(f_t \mid S_t = s),
\]
with $f_t$ drawn from the quantities estimated in Part I, including curvature $\kappa(t)$, 
curvature rate $\dot{\kappa}(t)$, speed $\|v(t)\|$, radial boundedness indicators, and derivative 
signatures.

Informally:
\begin{itemize}
    \item Straight motion ($S$) favors $\kappa(t) \approx 0$ with small $\dot{\kappa}(t)$.
    \item Turning motion ($T$) favors sustained nonzero curvature with sign consistency.
    \item Hovering ($H$) yields low speeds and unstable curvature estimates.
    \item Orb motion ($O$) favors smooth, bounded curvature with coherent evolution over time.
\end{itemize}

These likelihood models do not enforce deterministic boundaries; instead, they assign weights 
reflecting how well each state explains the observed geometry.

\paragraph{Physical and Dynamical Constraints.}
The state model incorporates minimal structural assumptions motivated by physical feasibility:
\begin{itemize}
    \item Hover ($H$) cannot transition abruptly to high-speed turning without intermediate 
          acceleration.
    \item Orb motion ($O$) requires multiple consecutive frames of bounded radial displacement.
    \item Straight and turn states ($S$ and $T$) may interleave but typically preserve short-term 
          curvature trends.
    \item State changes must occur over time scales longer than the sampling interval, unless 
          curvature or velocity exhibit clear discontinuities.
\end{itemize}

These constraints improve robustness by suppressing implausible sequences caused by noise or 
momentary reconstruction artifacts.

\paragraph{Dynamical Interpretation.}
The discrete state process provides an interpretable temporal summary of the trajectory. Rather 
than treating each frame independently, the Markov model links adjacent estimates, promoting 
temporal coherence and enabling identification of multi-frame regimes such as:
\begin{itemize}
    \item extended straight-line traversals,
    \item arcs or turning maneuvers,
    \item hovering intervals,
    \item sustained orbital patterns.
\end{itemize}

This structure forms the backbone of the unified classification framework. By encoding motion 
behavior as a sequence of discrete, interpretable states, the model creates a bridge between raw 
kinematic measurements and the higher-level dynamical diagnostics developed throughout Part II.

% --------------------------------
% 2.4 Observation Model
% --------------------------------

\subsection{Observation Model}
\label{subsec:observation_model}

The unified framework relies on geometric quantities—such as curvature, curvature rate, velocity, 
and radial displacement—that are derived from visual observations rather than direct physical 
measurements. As a result, the accuracy and interpretability of the model depend critically on the 
observation pipeline that reconstructs the trajectory $x(t)$ from the underlying sensor data. This 
section formalizes the assumptions and procedures that govern this reconstruction.

\paragraph{Platform-Motion Compensation.}
Raw sensor imagery often reflects the combined motion of the target and the recording platform. 
To isolate the target's apparent motion in the image plane, we apply stabilization procedures that 
compensate for platform drift, gimbal rotation, and camera jitter. Although the PR-018 dataset lacks 
complete metadata, frame-to-frame alignment via optical or structural features provides an 
approximate but effective means of isolating relative motion. The resulting stabilized track serves 
as the input for curvature and derivative estimation.

\paragraph{Noise Characterization.}
Visual tracking is inherently noisy due to factors including:
\begin{itemize}
    \item optical-flow estimation error,
    \item thermal noise and sensor quantization,
    \item compression artifacts,
    \item partial occlusions or frame cropping.
\end{itemize}

We treat these noise sources as zero-mean disturbances that perturb the observed position. Their 
primary effect is to introduce short-scale variability in velocity and second derivatives. Because 
curvature depends on these derivatives, subsequent smoothing is essential to recovering meaningful 
geometric structure.

\paragraph{Finite-Difference Derivative Estimation.}
Velocity and acceleration are estimated using centered finite differences over the stabilized 
trajectory:
\[
    v(t) \approx \frac{x(t+\Delta t) - x(t-\Delta t)}{2\Delta t}, \qquad
    a(t) \approx \frac{x(t+\Delta t) - 2x(t) + x(t-\Delta t)}{\Delta t^2}.
\]
These estimates define the curvature $\kappa(t)$ and curvature rate $\dot{\kappa}(t)$ introduced 
in Sections~\ref{subsec:geometry_overview}–\ref{subsec:curvature_state}. Finite differences are 
sensitive to noise but provide unbiased estimates under mild smoothness assumptions. Their 
simplicity also ensures reproducibility and transparency across datasets.

\paragraph{Temporal Smoothing of Geometric Quantities.}
To mitigate derivative amplification of noise, we apply temporal smoothing to $x(t)$ or to the 
derived velocity and curvature sequences. Smoothing may be implemented using moving averages, 
Savitzky–Golay filtering, or low-order polynomial fits, depending on dataset resolution. The 
objective is to preserve low-frequency geometric structure—such as sustained curvature or gradual 
turning—while suppressing frame-level jitter.

\paragraph{Mapping Observations to State Features.}
The observation model defines the sequence of feature vectors
\[
    f_t = \big(\kappa(t),\, \dot{\kappa}(t),\, \|v(t)\|,\, \text{radial}(t),\, a(t),\, j(t) \big),
\]
where “radial” denotes boundedness indicators and $j(t)$ the estimated jerk. These features serve as 
inputs to the likelihood models of Section~\ref{subsec:state_model}. Importantly, the observation model 
does not assume perfect accuracy: instead, it is designed to generate stable geometric summaries 
that remain robust under moderate noise.

\paragraph{Interpretive Role.}
By formalizing how raw sensor data is transformed into geometric descriptors, the observation model 
provides a transparent foundation for the unified framework. It clarifies the assumptions underlying 
trajectory reconstruction, identifies sources of uncertainty, and ensures that curvature-based 
classification remains grounded in observable quantities rather than unverified metadata or 
assumptions about sensor configuration.

% --------------------------------
% 2.5 Orb-State Justification
% --------------------------------

\subsection{Orb-State Justification}
\label{subsec:orb_justification}

The introduction of the orb state is motivated by empirical patterns in curvature and radial 
structure that cannot be adequately explained by classical categories such as straight motion, 
turning, or hovering. The orb state captures a regime in which the trajectory exhibits smooth, 
continuous curvature evolution together with bounded radial displacement relative to a local 
center. This combination of properties produces a distinctive dynamical signature that persists 
across multiple frames.

\paragraph{Curvature Evolution.}
Unlike standard turning motion—which is characterized by approximately constant curvature over a 
segment—the orb regime features curvature that varies continuously while remaining bounded away 
from zero. This produces a smooth “wavelike” evolution of $\kappa(t)$ and $\dot{\kappa}(t)$, 
reflecting gradual reorientation rather than rigid circular turning or erratic curvature noise. 
Empirically, such curvature traces display:
\begin{itemize}
    \item sustained nonzero curvature,
    \item coherent temporal evolution with low jerk,
    \item inflection patterns inconsistent with constant-radius turns,
    \item resilience to frame-level noise even after smoothing.
\end{itemize}

These features cannot be reliably modeled as either straight or turning motion.

\paragraph{Radial Boundedness.}
A defining characteristic of orbital-like motion is approximate radial boundedness: the trajectory
remains within a finite neighborhood of a time-varying local center. This condition is weaker than 
true circular motion—no strict radius or periodicity is assumed—but it distinguishes the orb state 
from turning, which typically lacks consistent radial structure. Radial boundedness is estimated 
using:
\begin{itemize}
    \item local center-of-curvature approximations,
    \item short-horizon estimates of displacement from inferred centers,
    \item variance thresholds on radial deviation.
\end{itemize}

When curvature is smooth and the radius of curvature fluctuates within bounded limits, the 
trajectory exhibits a recognizable orbital pattern.

\paragraph{Distinction From Hovering and Noise-Dominated Motion.}
Hovering motion often produces unstable curvature estimates because small positional noise creates 
large derivative fluctuations. In contrast, the orb state maintains stable curvature even at slow 
speeds due to coherent geometric structure. The orb state therefore cannot be explained as a 
noise artifact or jitter amplification and instead reflects meaningful kinematic organization.

\paragraph{Physical Non-Commitment.}
The orb state is not intended to imply any specific propulsion mechanism, control system, or 
underlying physics. It is a descriptive motion category defined solely by geometric and temporal 
signatures. The framework does not infer intent, internal structure, or energetic constraints; it 
merely classifies observable trajectory features into a consistent state taxonomy.

\paragraph{Diagnostic Value.}
Introducing the orb state improves classification accuracy and interpretability by preventing 
mis-labeling of coherent, bounded-curvature regimes as either turning or hovering. This yields:
\begin{itemize}
    \item cleaner temporal segmentation,
    \item reduced state ambiguity,
    \item improved likelihood separation,
    \item more informative downstream inference.
\end{itemize}

The orb state therefore plays a central role in the unified framework by capturing a distinct,
empirically motivated mode of motion that emerges naturally from curvature-based analysis.

% --------------------------------
% 2.6 Unified Likelihood Model
% --------------------------------

\subsection{Unified Likelihood Model}
\label{subsec:likelihood_model}

The unified likelihood model integrates geometric features, temporal dependencies, and state 
priors into a coherent probabilistic framework for motion-state classification. Given the feature 
sequence $\{f_t\}$ extracted from the observation model, the goal is to infer the posterior 
probabilities
\[
    \mathbb{P}(S_t = s \mid f_{1:t})
\]
for each state $s \in \mathcal{S}$. This section formalizes the likelihood structure that links 
geometric observations to the discrete state process.

\paragraph{Feature Likelihoods.}
For each state $s$, we define a state-conditional likelihood
\[
    \mathcal{L}(f_t \mid S_t = s),
\]
which evaluates how well the observed geometric features at time $t$ agree with the characteristic 
patterns of state $s$. These features include:
\begin{itemize}
    \item curvature $\kappa(t)$,
    \item curvature rate $\dot{\kappa}(t)$,
    \item speed $\|v(t)\|$,
    \item acceleration magnitude $\|a(t)\|$,
    \item radial boundedness indicators,
    \item jerk estimates $j(t)$.
\end{itemize}

The likelihood models are intentionally simple—typically Gaussian or log-normal components—
ensuring interpretability and robustness across heterogeneous datasets.

\paragraph{State Priors and Transition Dynamics.}
The Markovian structure introduced in Section~\ref{subsec:state_model} contributes a prior 
distribution on state evolution:
\[
    \mathbb{P}(S_t = s' \mid S_{t-1} = s) = T_{s,s'},
\]
where $T$ is the transition matrix. The transition priors encode temporal smoothness by 
favoring:
\begin{itemize}
    \item self-transitions (persistence of motion regimes),
    \item transitions between compatible states (e.g., straight-to-turn),
    \item rare transitions into or out of the orb state without geometric justification.
\end{itemize}

These priors regularize the classification process, reducing sensitivity to noise and preventing 
implausible state-switching behavior.

\paragraph{Joint Likelihood and Posterior Inference.}
Combining likelihoods and priors yields the joint probability of the state sequence and feature 
sequence:
\[
    \mathbb{P}(S_{1:T}, f_{1:T}) 
    = \mathbb{P}(S_1) \prod_{t=2}^T T_{S_{t-1},S_t}
      \prod_{t=1}^T \mathcal{L}(f_t \mid S_t).
\]

Posterior inference proceeds by computing:
\[
    \mathbb{P}(S_t \mid f_{1:T}),
\]
using standard dynamic programming techniques such as the forward–backward algorithm. This ensures 
that every state assignment reflects both local geometric evidence and global temporal coherence.

\paragraph{Interpretability of Posterior Probabilities.}
Posterior state probabilities provide an intuitive and transparent summary of the classifier's 
output. They allow researchers to:
\begin{itemize}
    \item identify dominant motion regimes over time,
    \item visualize confidence in state transitions,
    \item localize ambiguous or noisy segments,
    \item compare dynamical behavior across datasets.
\end{itemize}

Rather than producing a single hard label, the classifier outputs a probability distribution over 
states at each frame, allowing uncertainty to be represented explicitly.

\paragraph{Unified Structure.}
The unified likelihood model links together the major components of the theoretical framework:
\begin{itemize}
    \item curvature-based geometry (Sections~\ref{subsec:geometry_overview}--\ref{subsec:curvature_state}),
    \item temporal state dynamics (Section~\ref{subsec:state_model}),
    \item observation and noise modeling (Section~\ref{subsec:observation_model}),
    \item empirical justification for the orb regime (Section~\ref{subsec:orb_justification}).
\end{itemize}

This integration yields a physically grounded, statistically coherent model capable of describing 
unknown aerial trajectories with interpretable structure and quantifiable uncertainty.

% --------------------------------
% 2.7 Generalization Across Datasets
% --------------------------------

\subsection{Generalization Across Datasets}
\label{subsec:generalization}

A key objective of the unified framework is to provide consistent, interpretable motion-state 
classification across heterogeneous datasets. Although different sensors yield trajectories with 
varying resolutions, noise characteristics, and metadata completeness, the core geometric structure 
of the model remains invariant. This section outlines the elements of the framework that generalize 
across datasets and the adjustments required for practical application.

\paragraph{Geometric Invariance of Curvature-Based Features.}
Curvature, curvature rate, and radial structure are intrinsic properties of the trajectory and do 
not depend on absolute position, sensor calibration, or platform-specific metadata. As a result:
\begin{itemize}
    \item curvature-based motion signatures are comparable across sensors,
    \item feature extraction remains stable even when range or altitude are unknown,
    \item local geometric patterns provide a common basis for classification.
\end{itemize}

This invariance enables coherent interpretation of motion states despite differences in imaging 
modalities or acquisition conditions.

\paragraph{Robustness to Missing Metadata.}
Many publicly released sensor products lack complete information about focal length, gimbal 
calibration, frame timing, or platform motion. The framework accommodates such gaps by relying on:
\begin{itemize}
    \item relative motion rather than absolute physical units,
    \item smoothing and derivative estimation that operate purely on image-plane coordinates,
    \item likelihood models built on dimensionless geometric features.
\end{itemize}

This ensures that the classifier remains functional and interpretable even when only stabilized 
frame sequences are available, as in the PR-018 and GIMBAL datasets.

\paragraph{Dataset-Specific Noise Profiles.}
Different sensors introduce distinct noise characteristics—thermal noise in infrared imagery, 
compression artifacts in digital video, or tracking jitter in parallax-limited sequences. The 
observation model accommodates such variations through:
\begin{itemize}
    \item dataset-specific smoothing parameters,
    \item adaptive thresholds for curvature stability,
    \item variance-based weighting of derivative estimates.
\end{itemize}

Because the underlying geometric features are stable, only minimal adjustments to smoothing or 
noise filtering are required for each dataset.

\paragraph{State Interpretation Consistency.}
The state definitions introduced in Section~\ref{subsec:geometry_overview} remain valid across 
datasets. Straight, turn, hover, and orb states correspond to universal geometric regimes of 
trajectory evolution, independent of sensor modality or environmental conditions. As a result:
\begin{itemize}
    \item state sequences can be compared across datasets,
    \item posterior probabilities reflect consistent dynamical signatures,
    \item transitions carry the same interpretive meaning regardless of sensor source.
\end{itemize}

This consistency is especially important for evaluating whether different datasets exhibit 
qualitatively similar motion behavior.

\paragraph{Application to Multi-Sensor Cases.}
In sequences such as GIMBAL, parallax, platform rotation, and partial stabilization introduce 
additional geometric distortions. While these distortions affect the absolute shape of the 
trajectory, the unified framework remains applicable because:
\begin{itemize}
    \item the model uses local geometric features rather than global trajectory shape,
    \item curvature and curvature rate remain informative after smoothing,
    \item bounded-curvature regimes (orb states) manifest even under partial stabilization.
\end{itemize}

Dataset-specific corrections—such as pre-filtering or parallax compensation—can refine the 
trajectory but are not required for the core classifier to operate.

\paragraph{Summary.}
The unified framework generalizes across datasets because it is grounded in geometric features 
that are invariant under changes in sensor configuration, platform motion, and metadata 
availability. This flexibility supports consistent dynamical interpretation across PR-018, GIMBAL, 
and future datasets with similar characteristics.

% --------------------------------
% 2.8 Implications for Motion Interpretation
% --------------------------------

\subsection{Implications for Motion Interpretation}
\label{subsec:implications}

The unified framework provides a principled foundation for interpreting unknown aerial trajectories 
strictly in terms of observable geometric and dynamical structure. By grounding the analysis in 
curvature, state transitions, and probabilistic inference, the classifier avoids assumptions about 
propulsion, intent, or underlying mechanisms. This section summarizes the interpretive implications 
of the model and clarifies what conclusions may—and may not—be drawn from state sequences.

\paragraph{Constraints on Kinematic Hypotheses.}
Each motion state imposes specific geometric and dynamical constraints on feasible trajectories:
\begin{itemize}
    \item Straight segments ($S$) indicate persistent directional motion with minimal reorientation.
    \item Turning segments ($T$) reflect coordinated changes in direction with sustained curvature.
    \item Hover intervals ($H$) imply low-speed, small-displacement regimes.
    \item Orb segments ($O$) denote smooth, bounded-curvature evolution consistent with motion 
          around a local center.
\end{itemize}

These constraints allow researchers to evaluate which classes of kinematic models—ballistic, 
aerodynamic, noise-driven, or otherwise—are compatible with the observed state sequence.

\paragraph{Avoiding Overinterpretation.}
The classifier is not designed to infer causality or mechanism. For example:
\begin{itemize}
    \item an orb segment does not imply controlled flight or intent,
    \item a straight segment does not imply ballistic motion,
    \item a hovering interval does not imply levitation.
\end{itemize}

Instead, the classifier describes the *form* of motion, not its *origin*. Interpretation beyond 
kinematic structure requires additional physical information not present in image-plane trajectories.

\paragraph{State Sequences as Dynamical Summaries.}
State sequences serve as compact representations of the trajectory, enabling:
\begin{itemize}
    \item identification of dominant motion regimes,
    \item segmentation of complex trajectories into interpretable units,
    \item comparison of dynamical patterns across datasets,
    \item localization of anomalous or transitional intervals.
\end{itemize}

Posterior probabilities further highlight periods of ambiguity or uncertainty, allowing careful 
evaluation of borderline or noise-dominated segments.

\paragraph{Comparative Analysis Across Datasets.}
Because the states are defined geometrically, their interpretation remains consistent when applied 
to different sensor products. A sequence classified as $O \rightarrow T \rightarrow S$ carries the 
same dynamical meaning in PR-018 as in GIMBAL, even if the underlying sensor geometries differ. 
This supports cross-dataset comparison and strengthens the generality of the framework.

\paragraph{Limitations and Scope.}
The unified model is intentionally conservative. It operates entirely within the observational 
limits of image-plane trajectories and does not rely on assumptions about range, altitude, 
propulsion, or mass. As such:
\begin{itemize}
    \item physical inference beyond geometry is out of scope,
    \item energetic modeling is not attempted,
    \item mechanistic interpretation is avoided.
\end{itemize}

The framework is therefore most valuable as a diagnostic and comparative tool, not as a complete 
physical model of aerial motion.

\paragraph{Summary.}
The implications of the unified framework lie in its ability to provide repeatable, interpretable, 
and dataset-agnostic descriptions of motion behavior. By separating geometric observation from 
physical speculation, the model establishes a rigorous foundation for analyzing unknown aerial 
trajectories and preparing them for further scientific study.


\clearpage

% ============================
% PART III – MOTION-STATE CLASSIFIER
% ============================

\section*{Part III: Motion-State Classifier}
\addcontentsline{toc}{section}{Part III: Motion-State Classifier}
\label{sec:part3_classifier}
\section{PR018 Deep Analysis}
\label{sec:pr018_analysis}

This part applies the unified theoretical framework to the PR018 dataset, demonstrating how the 
classifier operates on real-world sensor data and evaluating the resulting dynamical-state 
interpretations.

\subsection{Dataset Overview and Reconstruction Summary}
\label{subsec:dataset_overview}

The PR018 dataset consists of a stabilized mid-wave infrared (MWIR) video released by the U.S. 
Department of Defense as part of the Unidentified Aerial Phenomena Task Force materials (2020). 
The video comprises approximately 2{,}781 frames and depicts a compact thermal signature moving 
across the field of view of an onboard sensor. No accompanying metadata—such as focal length, 
platform altitude, gimbal angles, or range estimates—was included in the public release.

Despite the absence of physical calibration parameters, the video exhibits several properties that 
enable geometric reconstruction of the object's apparent motion in the image plane:
\begin{itemize}
    \item the imagery is approximately stabilized around the target,
    \item frame timing appears consistent with a nominal $\sim 30$\,Hz sample rate,
    \item the target remains thermally saturated relative to the background,
    \item no major occlusions or dropouts occur within the analyzed interval.
\end{itemize}

\paragraph{Trajectory Extraction.}
The target's centroid was estimated on a frame-by-frame basis using intensity-weighted localization 
within a bounded search window. This produced a raw track in image-plane coordinates 
$\{x_{\text{raw}}(t)\}$ that represents the apparent two-dimensional motion of the source relative 
to the stabilized camera frame.

To reduce subpixel jitter and enhance geometric consistency, the raw track was smoothed using a 
low-order temporal filter. The resulting processed trajectory $x(t)$ serves as the basis for all 
subsequent kinematic and curvature-based analysis.

\paragraph{Derivative and Feature Estimation.}
Velocity, acceleration, curvature, and curvature rate were computed from $x(t)$ using the 
finite-difference and smoothing procedures described in Part~I and formalized in 
Section~\ref{subsec:observation_model}. These operations yielded the geometric feature sequence
\[
    f_t = \big(\kappa(t),\, \dot{\kappa}(t),\, \|v(t)\|,\, \text{radial}(t),\, a(t),\, j(t) \big),
\]
which forms the input to the unified likelihood model in Part~II.

\paragraph{Uncertainties and Assumptions.}
Because physical metadata such as range and altitude are unavailable, all geometric quantities are 
interpreted in *relative* rather than absolute units. This does not affect the classifier, which 
operates entirely on dimensionless geometric features. However, it does imply:
\begin{itemize}
    \item no claims about physical speed or distance can be made,
    \item only *shape* and *structure* of motion are analyzed,
    \item dynamical interpretations pertain to image-plane geometry.
\end{itemize}

The analysis therefore focuses on reproducible geometric signatures rather than on physical 
modeling or energetic estimation.

\paragraph{Purpose of the Dataset in This Framework.}
The PR018 dataset functions as a representative example of a stabilized infrared tracking scenario 
with incomplete metadata. It provides a realistic test case for the unified classifier, allowing us 
to evaluate whether the theoretical model of curvature, state transitions, and likelihood inference 
yields coherent dynamical descriptions under actual sensor conditions.

\subsection{Geometric Feature Extraction}
\label{subsec:feature_extraction}

The unified motion-state framework operates on geometric features derived from the processed 
trajectory $x(t)$. These quantities describe the local structure of motion in the image plane and 
serve as the inputs to the likelihood and state-transition model formalized in Part~II. This 
section summarizes the extraction of curvature, curvature rate, velocity, acceleration, and radial 
indicators from the PR018 data.

\paragraph{Velocity and Acceleration.}
Frame-to-frame velocities were computed using smoothed finite differences of the trajectory:
\[
v(t) = \frac{x(t+\Delta t) - x(t)}{\Delta t}.
\]
Second differences yield the acceleration estimate $a(t)$. Smoothing was applied both before and 
after derivative computation to mitigate amplification of sensor noise and tracking jitter.

\paragraph{Curvature.}
Curvature $\kappa(t)$ provides a scale-invariant measure of instantaneous trajectory bending:
\[
\kappa(t) = \frac{|x'(t) y''(t) - y'(t) x''(t)|}
                 {\big(x'(t)^2 + y'(t)^2 \big)^{3/2}}.
\]

\paragraph{Curvature Rate.}
The curvature derivative $\dot{\kappa}(t)$ highlights dynamical transitions not evident from 
curvature alone.

\paragraph{Radial Motion Indicators.}
We evaluate whether the trajectory exhibits bounded motion around a local center via
\[
r(t) = \| x(t) - c(t) \|,
\]
where $c(t)$ is a locally estimated curvature center.

\paragraph{Summary of Extracted Features.}
The resulting feature sequence
\[
f_t = \left(\kappa(t), \dot{\kappa}(t), \|v(t)\|, a(t), r(t)\right)
\]
forms a geometrically consistent description of the motion.

\subsection{State Classification Results}
\label{subsec:state_results}

Using the geometric feature sequence from 
Section~\ref{subsec:feature_extraction}, the unified likelihood model produced per-frame posterior 
probabilities:
\[
p_t = \big( p_t(S),\, p_t(T),\, p_t(H),\, p_t(O) \big).
\]

\paragraph{Dominant Orb Classification.}
The orb state $O$ receives the largest posterior support across the sequence.

\paragraph{Secondary Turn Segments.}
Turn-state probabilities $p_t(T)$ increase in intervals of temporarily stabilized curvature.

\paragraph{Limited Straight or Hover Support.}
Straight and hover states receive negligible posterior support.

\paragraph{Posterior Structure.}
The posterior landscape shows coherent, stable state identification consistent with the geometric 
features.

\subsection{Posterior Probability Interpretation}
\label{subsec:posterior_interpretation}

High-confidence orb intervals persist throughout most of the sequence. Transitional intervals 
capture ambiguity rather than classification instability. The posterior landscape remains stable 
across smoothing parameters.

\subsection{Consistency With Unified Framework}
\label{subsec:framework_consistency}

The PR018 results align strongly with all theoretical expectations of the unified framework.

\subsection{Summary of PR018 Analysis}
\label{subsec:pr018_summary}

The PR018 dataset demonstrates that the unified classifier provides a stable, interpretable, and 
geometrically grounded description of the observed motion. Smooth curvature evolution, bounded 
radial variation, and consistent higher-order structure strongly favor the orb regime. Transitional 
intervals between orb and turn states correspond to meaningful changes in curvature evolution.

These results serve two primary roles:
\begin{itemize}
    \item establishing a concrete demonstration of the unified framework under realistic sensor 
          conditions, and
    \item providing a reference pattern for comparison to other datasets such as GIMBAL.
\end{itemize}

The PR018 and GIMBAL analyses together outline the operational boundaries of the unified 
motion-state framework, demonstrating both its strengths and its appropriate limits.


To ensure that identified motion-state transitions reflect observable structure rather than estimator artifacts, we apply a small set of non-causal robustness checks summarized in Appendix~\ref{app:robustness}.

\clearpage

% ======================================
% PART IV – PR-018 DEEP ANALYSIS
% ======================================

\section*{Part IV: PR-018 Deep Analysis}
\addcontentsline{toc}{section}{Part IV: PR-018 Deep Analysis}
\input{manuscripts/Part_IV_PR018_Deep_Analysis/part4}

\clearpage

% =======================
% Part V — Unified Framework
% =======================
\input{manuscripts/Part_V_Unified_Framework/part5}

\clearpage

% ============================================================
% Part VI-A: Biological Migration Extension
% ============================================================

\section{Part VI-A: Biological Migration Extension}
\label{sec:part-vi-a}
\addcontentsline{toc}{section}{Part VI-A: Biological Migration Extension}

\section{Application of Structured Orb Dynamics to Biological Migration Systems}
\subsection{VI-A.1 Scope and Intent}

This section explores whether the geometry-first, state-based framework of Structured Orb Dynamics (SOD) generalizes to biological migration systems when observed under partial and uncertain measurement conditions. The objective is not to interpret biological behavior, infer intent, or model navigational mechanisms, but rather to evaluate the descriptive coherence of the SOD state machinery when applied outside the context of UAP trajectory analysis.

Migration trajectories present a well-studied yet inference-constrained class of motion characterized by long-range displacement, intermittent observation, environmental perturbation, and variable sampling resolution. These properties make migration an appropriate testbed for assessing whether SOD’s kinematic state definitions and segmentation logic remain stable under known, non-anomalous conditions.

No claims regarding causation, optimization, evolutionary strategy, or biological decision-making are made in this section. The analysis is strictly descriptive and methodological in scope.

\subsection{VI-A.2 Migration as a Motion-Inference Problem}

From a kinematic perspective, biological migration can be treated as a motion-inference problem involving extended trajectories observed incompletely across time and space. Tracking data are often sparse, irregularly sampled, and subject to environmental interference, occlusion, or measurement uncertainty. Despite these limitations, migration datasets nonetheless exhibit structured movement patterns across multiple temporal and spatial scales.

This section adopts a geometry-first framing, treating migratory paths as sequences of observed positions without embedding domain-specific assumptions regarding motivation, navigation cues, or biological purpose. By abstracting migration in this way, the analysis remains agnostic to species-level differences and focuses solely on trajectory geometry and state transitions observable in the data.

This framing mirrors the constraints encountered in UAP footage analysis, albeit within a domain where the class of moving objects is known. Migration therefore serves as a controlled reference domain for evaluating the stability and generality of the SOD framework.

\subsection{VI-A.3 Mapping SOD Motion States to Migratory Trajectories}

The SOD framework defines motion in terms of discrete kinematic states inferred from trajectory geometry. These states may be mapped to migratory trajectories without introducing biological interpretation, as described below:

Straight: Sustained directional travel characterized by low curvature variance and consistent displacement over time.

Turn: Reorientation events marked by localized increases in curvature, corresponding to course corrections or directional changes.

Hover: Localized motion with minimal net displacement, consistent with stopover behavior or confined exploration.

Orb: A kinematic regime dominated by low velocity, limited displacement, or elevated observational uncertainty.

In this context, the Orb state denotes a descriptive regime defined by motion ambiguity or measurement limits rather than a behavioral or biological classification. Its inclusion preserves consistency with earlier sections while reinforcing the distinction between kinematic description and interpretive attribution.

\subsection{VI-A.4 Geometry-First Metrics and State Segmentation}

Trajectory segmentation within migration datasets may be performed using the same geometry-first metrics employed throughout SOD, including curvature-based time series, segment duration distributions, and state transition probabilities. These metrics are evaluated independently of species identity, environmental context, or assumed navigational strategy.

By applying identical segmentation logic across domains, SOD enables direct comparison between migration trajectories and other partially observed motion systems. This invariance is central to assessing whether the framework captures fundamental properties of motion geometry rather than domain-specific artifacts.

\subsection{VI-A.5 Illustrative Pipeline Using Migratory Bird GPS Data}

To illustrate the domain generality of the Structured Orb Dynamics (SOD) framework beyond UAP trajectory analysis, a conceptual application to biological migration is outlined using GPS-tracked migratory bird data from the Movebank repository. The datasets considered represent seasonal continental migration in passerine species, corresponding to small-to-medium migratory land birds observed over extended spatial and temporal scales.

In this pipeline, individual migration tracks are treated solely as sequences of observed positions under partial and uncertain sampling. No species-specific, ecological, or behavioral annotations are incorporated into the analysis. Trajectories are first projected into a common analysis space, after which geometry-based metrics are computed to characterize local curvature, displacement, and temporal structure.

State segmentation is then performed using the same kinematic criteria applied throughout the SOD framework, assigning segments to descriptive motion regimes (Straight, Turn, Hover, and Orb) based exclusively on trajectory geometry and uncertainty considerations. Resulting segmentations enable examination of state durations and transition structure without invoking biological intent or navigational mechanisms.

This conceptual example is presented to demonstrate methodological consistency and domain generality rather than to report empirical findings. Migration trajectories serve here as a known-class motion system against which the coherence and stability of the SOD segmentation machinery may be evaluated under realistic observation constraints.

\begin{figure}[t]
    \centering

    \begin{subfigure}[b]{0.32\textwidth}
        \centering
        \includegraphics[width=\textwidth]{figures/VI-A_Panel_A_Trajectory.png}
        \caption{Projected trajectory}
        \label{fig:vi-a-panel-a}
    \end{subfigure}
    \hfill
    \begin{subfigure}[b]{0.32\textwidth}
        \centering
        \includegraphics[width=\textwidth]{figures/VI-A_Panel_B_Curvature.png}
        \caption{Curvature proxy vs.\ time}
        \label{fig:vi-a-panel-b}
    \end{subfigure}
    \hfill
    \begin{subfigure}[b]{0.32\textwidth}
        \centering
        \includegraphics[width=\textwidth]{figures/VI-A_Panel_C_Segmented_OptionB.png}
        \caption{State-segmented trajectory}
        \label{fig:vi-a-panel-c}
    \end{subfigure}

    \caption{Geometry-first state segmentation of a partially observed trajectory.
    (A) Observed trajectory projected into analysis space without domain-specific assumptions regarding object class, intent, or environmental context.
    (B) Curvature time series derived from the projected trajectory, illustrating regions of sustained directional motion, localized reorientation, and intervals where sparse sampling increases uncertainty.
    (C) Trajectory segmented into Structured Orb Dynamics (SOD) kinematic states using geometry-based criteria. Low-displacement segments that are identifiable under sufficient temporal separation are classified as Hover and treated as a secondary sub-case within the segmentation logic. Segments dominated by sampling gaps or ambiguous geometry are classified as \emph{Orb}, reflecting uncertainty rather than inferred behavior.
    The trajectory shown corresponds to a GPS-tracked migratory passerine bird from publicly available Movebank data and is included as an illustrative example of a known-class motion system under partial observation.}
    \label{fig:vi-a-sod-migration}
\end{figure}

\subsection{VI-A.6 Relation to UAP Trajectory Analysis}

The inclusion of biological migration as an exploratory extension does not alter the interpretive constraints applied to UAP footage. Instead, migration trajectories provide a known-class reference for evaluating state segmentation behavior under partial observation.

While the same kinematic machinery may be applied across domains, the interpretive layer remains domain-specific. Migration serves as a validation context for assessing segmentation stability, whereas UAP footage represents an extreme uncertainty regime where object class and intent are not assumed.

\subsection{VI-A.7 Limitations and Non-Claims}

This section does not attempt to infer biological intent, model navigation mechanisms, evaluate energetic optimization, or explain migratory origin or purpose. The analysis remains confined to kinematic description under uncertainty.

Similarly, no claims are made regarding the explanatory power of SOD beyond motion characterization. The framework is not proposed as a theory of behavior or causation.

\subsection{VI-A.8 Implications for General Motion Analysis}

Demonstrating that SOD applies coherently to biological migration supports its characterization as a general motion-inference framework rather than a domain-specific tool. This extension broadens the potential applicability of SOD while preserving falsifiability, interpretive restraint, and methodological clarity.


% legacy / deprecated results draft (tables) — keep out of build
% \input{manuscripts/Part_VI_A_Exploratory_Extension/part6a}

\input{manuscripts/Part_VI_A_Exploratory_Extension/part6b}
\input{manuscripts/Part_VI_A_Exploratory_Extension/part6c}

% ======================================
% SUPPLEMENTARY GIMBAL ANALYSIS
% ======================================

\section*{Supplementary GIMBAL Analysis}
\addcontentsline{toc}{section}{Supplementary GIMBAL Analysis}
% ==========================================
% Supplementary GIMBAL Analysis (Skeleton)
% ==========================================

\section{Supplementary Analysis: GIMBAL Dataset}
\label{sec:gimbal}

This supplementary section applies components of the unified motion-state framework to the 
GIMBAL dataset, illustrating how curvature-based geometric diagnostics behave under conditions 
where absolute trajectory reconstruction is not feasible due to platform-induced parallax and 
missing metadata.

\subsection{Dataset Characteristics and Limitations}
\label{subsec:gimbal_characteristics}

The GIMBAL dataset consists of a forward-looking infrared (FLIR) video recorded from a moving 
aircraft platform. The video shows a bright, extended infrared source that remains near the center 
of the field of view while the background and horizon rotate due to gimbal and platform motion. 
As with PR018, no complete physical metadata---such as focal length, range, altitude, gimbal 
angles, or inertial platform data---is provided in the publicly released version.

Several characteristics of the GIMBAL sequence distinguish it from PR018 and introduce important 
constraints on geometric interpretation:

\begin{itemize}
    \item \textbf{Strong platform-induced parallax:} The apparent motion of the background is 
    dominated by the aircraft's own maneuvering and gimbal rotation, making it difficult to 
    isolate the target's absolute motion in the image plane.

    \item \textbf{Persistent target centering:} The target remains close to the center of the frame, 
    indicating active tracking by the sensor. As a result, target displacement is largely absorbed 
    by gimbal motion rather than appearing as a large-scale trajectory across the image.

    \item \textbf{Partial stabilization and rotation:} The horizon and background features undergo 
    rotation and deformation that reflect a combination of platform motion, gimbal adjustments, and 
    possible zoom changes. This complicates any attempt to define a fixed image-plane reference 
    frame.

    \item \textbf{Incomplete telemetry and calibration:} Without synchronized platform, gimbal, and 
    range data, it is not possible to reconstruct a metric three-dimensional trajectory or to 
    separate parallax effects from intrinsic target motion.
\end{itemize}

\paragraph{Implications for Trajectory Reconstruction.}
In PR018, approximate stabilization around the target allowed the construction of a meaningful 
image-plane trajectory $x(t)$ suitable for curvature and state analysis. In contrast, the GIMBAL 
sequence does not admit a unique or reliable reconstruction of the target's intrinsic motion. Any 
attempt to infer a full trajectory would require strong assumptions about platform kinematics, 
gimbal control laws, and range, none of which are available in the public release.

As a consequence, the unified framework cannot be applied to GIMBAL in the same quantitative manner 
as it was to PR018. Instead, the analysis is necessarily more limited and qualitative, focusing on 
what can be inferred from:

\begin{itemize}
    \item relative orientation changes between the target and the horizon,
    \item local apparent motion of the target within the tracked window,
    \item structural stability of the target's image under platform rotation.
\end{itemize}

\paragraph{Scope of the Supplementary Analysis.}
Given these constraints, the GIMBAL analysis is framed as a supplementary, qualitative application 
of the unified motion-state framework. The goals are to:

\begin{itemize}
    \item identify which geometric cues are still accessible despite parallax and missing metadata,
    \item assess whether these cues are consistent with the motion-state taxonomy developed in 
          Part~II,
    \item compare qualitative dynamical patterns with those observed in PR018,
    \item avoid overinterpretation by explicitly acknowledging the limits of the available data.
\end{itemize}

No attempt is made to derive a full image-plane trajectory or to compute curvature with the same 
precision used for PR018. Instead, the GIMBAL dataset serves as a stress test of the framework's 
interpretive discipline under conditions where only partial geometric information is available.

\paragraph{Conservative Interpretive Stance.}
Throughout this supplementary analysis, the emphasis remains on descriptive geometry and 
qualitative consistency, not on physical or mechanistic inference. The limitations of the GIMBAL 
data are treated as a fundamental boundary on what can be responsibly concluded. Within that 
boundary, the unified framework provides a structured language for discussing observable motion 
patterns without exceeding the evidential content of the dataset.

\subsection{Extractable Geometric Features}
\label{subsec:gimbal_features}

Although the GIMBAL dataset does not permit reconstruction of an absolute image-plane trajectory, 
several geometric cues remain accessible and can be interpreted within the qualitative structure of 
the unified motion-state framework. These cues arise from the residual motion of the target within 
the tracked window, from the behavior of the horizon and background, and from the stability of the 
target’s infrared signature under platform rotation.

\paragraph{Local Apparent Motion of the Target.}
Even though the target is actively stabilized near the center of the frame, small residual 
displacements are visible over time. These residual motions cannot be converted into metric units 
or reliable curvature estimates, but they do provide information about:
\begin{itemize}
    \item slight lateral drift within the tracking window,
    \item transient deviations from perfect centering,
    \item micro-adjustments made by the tracking and gimbal-control system.
\end{itemize}
Such motions reflect a combination of sensor behavior and relative target motion, and are treated 
here strictly as qualitative indicators of smoothness or irregularity.

\paragraph{Horizon Rotation and Background Flow.}
The most prominent large-scale geometric feature in GIMBAL is the rotation of the horizon and 
background. This rotation is driven primarily by platform and gimbal motion. When viewed as a 
time-varying angle, the horizon provides:
\begin{itemize}
    \item a proxy for the aircraft’s maneuvering and gimbal rotation,
    \item a reference frame for evaluating the target’s stability relative to the rotating scene,
    \item a qualitative separation between platform-induced dynamics and target-centric motion.
\end{itemize}
The fact that the target remains centered during substantial horizon rotation suggests that its 
apparent position is governed largely by the tracking system rather than by large-scale motion 
across the field of view.

\paragraph{Image Stability and Shape Consistency.}
The target’s infrared signature retains a stable morphology throughout the sequence. This stability 
enables limited geometric inference:
\begin{itemize}
    \item the absence of pronounced shape deformation argues against high-acceleration image-plane motion,
    \item the lack of abrupt centroid jumps suggests no strong high-frequency oscillatory behavior,
    \item coherent intensity structure indicates that residual motions are relatively smooth.
\end{itemize}
These observations provide qualitative evidence that whatever apparent motion exists is not 
dominated by jitter or tracking failures.

\paragraph{Relative Angular Behavior.}
By examining the orientation of the target relative to the rotating horizon, one can extract 
qualitative information about relative angular behavior:
\begin{itemize}
    \item whether the target exhibits slow drift relative to the horizon,
    \item whether its apparent alignment changes gradually or abruptly,
    \item whether any relative angular behavior is consistent across distinct phases of platform rotation.
\end{itemize}
These relative angular cues do not yield curvature or trajectory estimates but help distinguish 
smooth, coordinated evolution from irregular or noise-driven changes.

\paragraph{Limitations of Extractable Features.}
Crucially, the available geometric cues do \emph{not} permit:
\begin{itemize}
    \item computation of curvature $\kappa(t)$ or curvature rate $\dot{\kappa}(t)$,
    \item estimation of a center-of-curvature or radial profile,
    \item recovery of absolute velocity, acceleration, or jerk,
    \item construction of a consistent image-plane trajectory comparable to PR018.
\end{itemize}
For this reason, the GIMBAL analysis remains strictly qualitative. All inferences are framed in 
terms of relative stability, smoothness, and alignment, rather than quantitative kinematic 
measures.

\paragraph{Role Within the Unified Framework.}
Despite these limitations, the extractable features serve an important role: they allow us to 
evaluate whether the observable behavior of the target is qualitatively compatible with the kinds 
of smooth, persistent regimes associated with orb-like or turn-like motion in the unified 
framework, or whether it appears inconsistent with those regimes. These cues provide the basis for 
the qualitative state-consistency assessment in the next subsection.

\subsection{Qualitative State-Consistency Assessment}
\label{subsec:gimbal_state_consistency}

Because the GIMBAL dataset does not permit extraction of a reliable image-plane trajectory, the 
state-consistency analysis cannot rely on quantitative curvature $\kappa(t)$, curvature rate 
$\dot{\kappa}(t)$, or finite-difference kinematic derivatives. Instead, we evaluate whether the 
observable geometric cues identified in Section~\ref{subsec:gimbal_features} exhibit qualitative 
patterns consistent with any of the motion-state regimes defined in the unified framework.

This assessment is interpretive rather than computational, and is guided by three criteria:
\begin{enumerate}
    \item structural smoothness of the target’s apparent behavior,
    \item temporal persistence or coherence across distinct intervals of the recording,
    \item compatibility with the qualitative signatures of the Straight, Turn, Hover, and Orb states.
\end{enumerate}
The goal is not to assign a discrete state sequence, but to evaluate whether any observable 
behaviors contradict or align with the taxonomy developed in Part~II.

\paragraph{Straight-State Consistency.}
A motion would be \emph{qualitatively straight-like} if the target exhibited:
\begin{itemize}
    \item minimal displacement within the field of view,
    \item negligible relative drift against the rotating horizon,
    \item image stability with no indication of curvature-driven lateral motion.
\end{itemize}
GIMBAL does show periods of limited displacement, but these coincide with active sensor tracking 
and cannot be interpreted as evidence of straight-line intrinsic motion. No inconsistencies with a 
straight-state pattern are observed, but no affirmative geometric evidence supports it either.

\paragraph{Turn-State Consistency.}
A qualitative turn-like regime would require:
\begin{itemize}
    \item consistent lateral displacement relative to a background reference,
    \item monotonic angular change or persistent deviation from straight-like behavior,
    \item residual motion incompatible with pure stabilization.
\end{itemize}
Because horizon rotation is dominated by platform motion, the dataset does not present reliable 
turn-like cues. No geometric feature in GIMBAL contradicts a turn-like interpretation, but none 
provides positive support for one.

\paragraph{Hover-State Consistency.}
Hover-like behavior would include:
\begin{itemize}
    \item minimal intrinsic displacement,
    \item absence of systematic drift,
    \item frame-to-frame stability suggestive of low-speed or stationary dynamics.
\end{itemize}
The target’s near-centering and morphological stability could appear hover-like, but these are 
fully attributable to the sensor tracking system. Thus, hover-like patterns are \emph{not} ruled 
out, but cannot be meaningfully inferred.

\paragraph{Orb-State Consistency.}
The orb state is characterized by smoothly evolving curvature and persistence of dynamical regime 
beyond classical aerodynamic expectations. Qualitatively, this would manifest as:
\begin{itemize}
    \item smooth, coordinated evolution relative to a rotating reference frame,
    \item absence of ballistic straight-like segments,
    \item no abrupt kinematic transitions or discontinuities.
\end{itemize}
Although curvature cannot be computed for GIMBAL, two observable behaviors are noteworthy:
\begin{enumerate}
    \item the target’s apparent motion (to the extent it is visible) shows no abrupt jumps,
    \item its morphological stability persists during substantial platform-induced rotation.
\end{enumerate}
These observations are \emph{consistent} with smooth evolution, but they do not distinguish between 
hover-like, turn-like, or orb-like regimes. The dataset lacks the geometric resolution needed to 
affirm or reject orb-state behavior.

\paragraph{Synthesis.}
Across all four states, we find:
\begin{itemize}
    \item no observable behavior that contradicts any state class outright,
    \item insufficient geometric information to positively support any single state,
    \item qualitative compatibility with multiple regimes due to the dominance of parallax and tracking.
\end{itemize}
Thus, the appropriate conclusion is one of disciplined uncertainty:  
\textit{the GIMBAL dataset does not provide enough geometric information to determine which motion 
state, if any, the target occupies}.  
This stands in contrast to PR018, where curvature-based analysis enabled quantitative state 
assignment.

The qualitative consistency analysis therefore reinforces the conservative stance taken throughout 
this supplement: the GIMBAL sequence lacks the evidential content required for state classification 
but remains fully interpretable within the unified motion-state framework.

\subsection{Comparison With PR018 Results}
\label{subsec:gimbal_comparison}

A central purpose of this supplementary analysis is to clarify how the qualitative cues observable 
in the GIMBAL dataset relate to the quantitative curvature-based results obtained from PR018. 
Because the two datasets differ fundamentally in the availability of stabilization, parallax 
conditions, and metadata, any comparison must explicitly separate \emph{what is comparable} from 
\emph{what is not}.

\paragraph{Differences in Data Structure.}
PR018 provides:
\begin{itemize}
    \item a stable, target-centered field of view,
    \item sufficient residual motion to reconstruct a 2D image-plane trajectory,
    \item conditions enabling computation of curvature, curvature rate, and finite-difference 
          kinematic derivatives.
\end{itemize}
By contrast, GIMBAL provides:
\begin{itemize}
    \item strong parallax dominated by platform and gimbal motion,
    \item tracking behavior that suppresses visible target displacement,
    \item no reliable reference frame for trajectory reconstruction.
\end{itemize}
As a result, the unified framework can be \emph{applied quantitatively} to PR018, but only 
\emph{qualitatively} to GIMBAL.

\paragraph{Comparability of Motion-State Signatures.}
The motion-state analysis in PR018 was driven by curvature signatures, state-likelihood evaluation, 
and posterior temporal structure. None of these quantities can be computed for GIMBAL. Therefore:
\begin{itemize}
    \item no direct comparison of curvature $\kappa(t)$ or curvature continuity can be made,
    \item no state-probability sequence can be constructed for GIMBAL,
    \item no dynamical inferences (straight, turn, hover, orb) can be quantitatively aligned.
\end{itemize}
The two datasets are therefore \emph{not comparable at the level of dynamical-state inference}.

\paragraph{Comparability of Qualitative Behaviors.}
Even without a trajectory, certain qualitative behaviors can still be contrasted. We highlight three 
points of relevance:

\begin{enumerate}
    \item \textbf{Smoothness vs. discontinuity.}  
    PR018 exhibited smooth curvature evolution with no surviving straight-line segments after 
    stabilization. GIMBAL shows no abrupt centroid jumps or discontinuous behavior, but this 
    smoothness may arise solely from active tracking.

    \item \textbf{Persistence of geometric structure.}  
    In PR018, the orb-like regime persisted across hundreds of frames and was supported by 
    quantitative diagnostics. In GIMBAL, the target’s morphological stability is persistent but 
    does not reflect intrinsic motion.

    \item \textbf{Relative-frame behavior.}  
    PR018 provided a stable image-plane frame that reflected target motion. In GIMBAL, the 
    rotating horizon dominates the frame, preventing any equivalence in geometric interpretation.
\end{enumerate}

While some qualitative superficial similarities exist—such as stable centroid structure and absence 
of jitter—they do not imply dynamical similarity.

\paragraph{Constraints on Physical Interpretation.}
In PR018, the unified framework permitted elimination of several physical categories due to the 
observed curvature behavior (e.g., no ballistic segments). No such eliminations are possible for 
GIMBAL. Specifically:
\begin{itemize}
    \item GIMBAL cannot rule out straight, turn, hover, or orb-like regimes,
    \item GIMBAL cannot constrain acceleration or curvature profiles,
    \item GIMBAL cannot be used to evaluate dynamical feasibility of hypotheses.
\end{itemize}
This distinction is crucial for scientific rigor: the evidential strength of the two datasets is 
fundamentally unequal.

\paragraph{Synthesis and Interpretation.}
The comparison yields three disciplined conclusions:

\begin{enumerate}
    \item \textbf{PR018 and GIMBAL cannot be treated as equivalent for motion-state analysis.}  
    Only PR018 offers the necessary geometric observables for quantitative classification.

    \item \textbf{Qualitative cues in GIMBAL do not contradict the unified framework.}  
    GIMBAL shows no behavior incompatible with smooth or persistent regimes, but this carries no 
    dynamical weight.

    \item \textbf{GIMBAL provides a test of interpretive discipline rather than motion inference.}  
    Its value lies in confirming that the unified framework maintains methodological restraint in 
    the presence of incomplete data.
\end{enumerate}

Thus, while PR018 supports a quantitative demonstration of the unified motion-state model, the 
GIMBAL dataset serves as a complementary case illustrating how the framework operates under strong 
observational constraints without exceeding the evidential content of the data.

\subsection{Summary of GIMBAL Supplement}
\label{subsec:gimbal_summary}

The GIMBAL dataset provides a valuable but fundamentally limited test case for the unified 
motion-state framework. Unlike PR018, where approximate stabilization and sufficient residual 
motion enabled quantitative curvature analysis, the GIMBAL sequence is dominated by 
platform-induced parallax, active tracking behavior, and a lack of supporting telemetry. These 
conditions preclude reconstruction of an image-plane trajectory and prevent computation of 
curvature, curvature rate, or state-likelihood estimates.

Despite these constraints, several meaningful conclusions emerge from the qualitative analysis:

\paragraph{1. Geometric cues are available but non-quantitative.}
Features such as horizon rotation, target morphological stability, and small residual centroid 
motions provide limited geometric insight. However, these cues cannot be used to infer intrinsic 
target motion, and no dynamical quantities can be estimated.

\paragraph{2. No state classification is possible.}
Because curvature-based diagnostics cannot be computed, none of the four motion states 
(straight, turn, hover, orb) can be positively identified or excluded.  
The dataset is consistent with all regimes but affirmatively supports none.

\paragraph{3. Qualitative behaviors do not contradict the unified framework.}
The observable properties of the GIMBAL target—morphological stability, lack of jitter, smooth 
appearance relative to a rotating frame—are compatible with the conceptual structure of the 
framework. Crucially, this compatibility arises from insufficient data rather than evidence of 
smooth or persistent motion.

\paragraph{4. Comparison with PR018 highlights a contrast in evidential strength.}
PR018 provides direct geometric observables enabling quantitative state inference and elimination 
of incompatible physical models.  
GIMBAL does not.  
This contrast illustrates the importance of dataset structure in motion-state classification.

\paragraph{5. The supplement demonstrates interpretive discipline.}
A central contribution of the unified motion-state framework is its insistence on proportional 
inference: conclusions must scale with the quantity and reliability of available geometric data.  
The GIMBAL supplement shows that the framework remains fully applicable—but appropriately 
restrained—even when only partial information is accessible.

\paragraph{Overall Synthesis.}
The GIMBAL dataset does not support quantitative dynamical modeling or state assignment, but it 
plays an important complementary role in the manuscript:

\begin{itemize}
    \item it tests the unified framework’s ability to operate under severe observational constraints,
    \item it highlights the methodological distinction between geometric visibility and interpretive certainty,
    \item it reinforces the evidential strength of the PR018 analysis by contrast,
    \item it establishes clear scientific boundaries that prevent over interpretation.
\end{itemize}

In this sense, the supplementary analysis satisfies a critical epistemic function:  
\emph{it demonstrates that the unified motion-state framework is robust not only when data are 
rich but also when they are incomplete—while maintaining fidelity to the principles of 
quantitative geometry and conservative inference.}




\clearpage

\section*{Author Contributions}
\addcontentsline{toc}{section}{Author Contributions}

Cassandra Perry: Conceptualization; methodology; data curation; geometric analysis; model
development; manuscript preparation; validation; software design; interpretation of results;
project administration.

ORCID: 0009-0001-1998-1481

% ======================================
% ACKNOWLEDGMENTS
% ======================================
\section*{Acknowledgments}
\addcontentsline{toc}{section}{Acknowledgments}

The author gratefully acknowledges the support of the AI research assistant 
``Eleanor,'' whose contributions in structuring mathematical arguments, refining 
methodological clarity, and assisting with manuscript organization facilitated 
the development of this unified framework. All conceptual decisions, 
interpretations, and conclusions are solely those of the author.

The author also thanks the open-science community for maintaining accessible 
archival infrastructure and acknowledges the value of public-domain sensor data 
that enables reproducible geometric analysis.

\clearpage

% ======================================
% DISCUSSION AND OUTLOOK
% ======================================

\section*{Discussion and Outlook}
\addcontentsline{toc}{section}{Discussion and Outlook}

The unified geometric motion-state framework introduced in this manuscript 
demonstrates that quantitative inference is possible even when physical metadata, 
telemetry, and calibration information are incomplete. By grounding the analysis 
in curvature, its temporal evolution, and dimensionless kinematic structure, the 
approach offers a reproducible method for describing unknown aerial trajectories 
in a manner that is conservative, interpretable, and consistent across 
heterogeneous datasets.

The PR-018 analysis shows that certain curvature regimes---most notably the 
smooth, bounded-curvature patterns associated with the orb state---persist across 
hundreds of frames and remain stable under reasonable choices of smoothing and 
derivative estimation. By contrast, the GIMBAL supplement illustrates how the 
same framework behaves under conditions where trajectory reconstruction is not 
possible. Taken together, these cases highlight the strengths, limitations, and 
future directions for geometric motion analysis in the absence of conventional 
metadata.

Future work will extend the classifier, refine curvature-based diagnostics, and 
explore methods for multi-sensor fusion when partial telemetry is available. The 
broader outlook is the development of a generalizable motion-analysis toolkit 
suitable for research domains where observational constraints are the norm rather 
than the exception.

\paragraph{Role of Known-Class Motion Systems.}
Part VI-A is included as an minimal empirical extension intended to examine the internal coherence of the Structured Orb Dynamics (SOD) framework when applied to a known-class motion system observed under partial and uncertain conditions. The migratory bird trajectories considered therein are not used to infer biological behavior, navigation strategy, or optimization, nor are they presented as validation benchmarks. Instead, they serve as a controlled reference domain for assessing whether the same geometry-first state definitions and segmentation logic employed in UAP trajectory analysis remain stable when object class is known but observational completeness is limited. This comparison clarifies the scope of SOD as a descriptive motion-inference framework rather than a domain-specific explanatory model.

\clearpage

% ======================================
% DATA AVAILABILITY
% ======================================

\section*{Data Availability}
\addcontentsline{toc}{section}{Data Availability}

All reconstructed trajectories, derived curvature products, classifier outputs, 
and analysis scripts used in this work are publicly available in a versioned 
Zenodo repository associated with this manuscript. The PR-018 and GIMBAL 
infrared videos analyzed herein are publicly released products of the U.S. 
Department of Defense.

\clearpage

% ======================================
% SOFTWARE AVAILABILITY
% ======================================

\section*{Software Availability}
\addcontentsline{toc}{section}{Software Availability}

The full implementation of the Orb Motion Classifier---including trajectory 
processing, curvature estimation, state-likelihood evaluation, and posterior 
inference---is available on GitHub under an open license. The repository 
provides complete documentation, reproducible scripts, and archived releases 
linked to Zenodo DOIs for long-term preservation.

\clearpage

\appendix
\section{Observable Robustness and Non-Causal Validation}
\label{app:robustness}

\noindent\textit{Status: protocol-ready.} The checks below define a non-causal robustness protocol intended for systematic application across datasets and figures, and may be applied selectively depending on data quality and availability.

\subsection{Track Continuity}
We verify continuity of the tracked trajectory across transition boundaries using an arc-length or speed proxy,
\[
\Delta s_t = \lVert \mathbf{x}_{t+1} - \mathbf{x}_t \rVert,
\]
to confirm that no discontinuities arise from sampling gaps, re-indexing, or tracking loss. Sustained continuity across the transition supports interpretation as a single, coherent track.

\subsection{Estimator Robustness}
Curvature-based segmentation is evaluated under modest methodological perturbations, including smoothing variation, temporal subsampling, and alternative discrete curvature estimators (e.g., turning-angle-based versus multi-point curvature). A transition window is considered robust if its presence and temporal ordering persist across these variations.

\subsection{Change-Point Detection}
To reduce reliance on manually selected thresholds, algorithmic change-point detection is applied to the curvature time series. Agreement between detected change points and segmentation boundaries supports algorithmic identification of transition intervals.

\subsection{Uncertainty and Baseline Distinction}
Baseline curvature variability is estimated for low-curvature segments to construct a simple uncertainty or confidence band. Transition intervals are evaluated for sustained deviation beyond baseline variability, distinguishing structured curvature elevation from noise fluctuations.

\clearpage
\bibliographystyle{ieeetr}
\bibliography{references}

\end{document}
