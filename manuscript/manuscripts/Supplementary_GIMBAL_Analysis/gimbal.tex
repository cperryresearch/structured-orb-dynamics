% ==========================================
% Supplementary GIMBAL Analysis (Skeleton)
% ==========================================

\section{Supplementary Analysis: GIMBAL Dataset}
\label{sec:gimbal}

This supplementary section applies components of the unified motion-state framework to the 
GIMBAL dataset, illustrating how curvature-based geometric diagnostics behave under conditions 
where absolute trajectory reconstruction is not feasible due to platform-induced parallax and 
missing metadata.

\subsection{Dataset Characteristics and Limitations}
\label{subsec:gimbal_characteristics}

The GIMBAL dataset consists of a forward-looking infrared (FLIR) video recorded from a moving 
aircraft platform. The video shows a bright, extended infrared source that remains near the center 
of the field of view while the background and horizon rotate due to gimbal and platform motion. 
As with PR018, no complete physical metadata---such as focal length, range, altitude, gimbal 
angles, or inertial platform data---is provided in the publicly released version.

Several characteristics of the GIMBAL sequence distinguish it from PR018 and introduce important 
constraints on geometric interpretation:

\begin{itemize}
    \item \textbf{Strong platform-induced parallax:} The apparent motion of the background is 
    dominated by the aircraft's own maneuvering and gimbal rotation, making it difficult to 
    isolate the target's absolute motion in the image plane.

    \item \textbf{Persistent target centering:} The target remains close to the center of the frame, 
    indicating active tracking by the sensor. As a result, target displacement is largely absorbed 
    by gimbal motion rather than appearing as a large-scale trajectory across the image.

    \item \textbf{Partial stabilization and rotation:} The horizon and background features undergo 
    rotation and deformation that reflect a combination of platform motion, gimbal adjustments, and 
    possible zoom changes. This complicates any attempt to define a fixed image-plane reference 
    frame.

    \item \textbf{Incomplete telemetry and calibration:} Without synchronized platform, gimbal, and 
    range data, it is not possible to reconstruct a metric three-dimensional trajectory or to 
    separate parallax effects from intrinsic target motion.
\end{itemize}

\paragraph{Implications for Trajectory Reconstruction.}
In PR018, approximate stabilization around the target allowed the construction of a meaningful 
image-plane trajectory $x(t)$ suitable for curvature and state analysis. In contrast, the GIMBAL 
sequence does not admit a unique or reliable reconstruction of the target's intrinsic motion. Any 
attempt to infer a full trajectory would require strong assumptions about platform kinematics, 
gimbal control laws, and range, none of which are available in the public release.

As a consequence, the unified framework cannot be applied to GIMBAL in the same quantitative manner 
as it was to PR018. Instead, the analysis is necessarily more limited and qualitative, focusing on 
what can be inferred from:

\begin{itemize}
    \item relative orientation changes between the target and the horizon,
    \item local apparent motion of the target within the tracked window,
    \item structural stability of the target's image under platform rotation.
\end{itemize}

\paragraph{Scope of the Supplementary Analysis.}
Given these constraints, the GIMBAL analysis is framed as a supplementary, qualitative application 
of the unified motion-state framework. The goals are to:

\begin{itemize}
    \item identify which geometric cues are still accessible despite parallax and missing metadata,
    \item assess whether these cues are consistent with the motion-state taxonomy developed in 
          Part~II,
    \item compare qualitative dynamical patterns with those observed in PR018,
    \item avoid overinterpretation by explicitly acknowledging the limits of the available data.
\end{itemize}

No attempt is made to derive a full image-plane trajectory or to compute curvature with the same 
precision used for PR018. Instead, the GIMBAL dataset serves as a stress test of the framework's 
interpretive discipline under conditions where only partial geometric information is available.

\paragraph{Conservative Interpretive Stance.}
Throughout this supplementary analysis, the emphasis remains on descriptive geometry and 
qualitative consistency, not on physical or mechanistic inference. The limitations of the GIMBAL 
data are treated as a fundamental boundary on what can be responsibly concluded. Within that 
boundary, the unified framework provides a structured language for discussing observable motion 
patterns without exceeding the evidential content of the dataset.

\subsection{Extractable Geometric Features}
\label{subsec:gimbal_features}

Although the GIMBAL dataset does not permit reconstruction of an absolute image-plane trajectory, 
several geometric cues remain accessible and can be interpreted within the qualitative structure of 
the unified motion-state framework. These cues arise from the residual motion of the target within 
the tracked window, from the behavior of the horizon and background, and from the stability of the 
target’s infrared signature under platform rotation.

\paragraph{Local Apparent Motion of the Target.}
Even though the target is actively stabilized near the center of the frame, small residual 
displacements are visible over time. These residual motions cannot be converted into metric units 
or reliable curvature estimates, but they do provide information about:
\begin{itemize}
    \item slight lateral drift within the tracking window,
    \item transient deviations from perfect centering,
    \item micro-adjustments made by the tracking and gimbal-control system.
\end{itemize}
Such motions reflect a combination of sensor behavior and relative target motion, and are treated 
here strictly as qualitative indicators of smoothness or irregularity.

\paragraph{Horizon Rotation and Background Flow.}
The most prominent large-scale geometric feature in GIMBAL is the rotation of the horizon and 
background. This rotation is driven primarily by platform and gimbal motion. When viewed as a 
time-varying angle, the horizon provides:
\begin{itemize}
    \item a proxy for the aircraft’s maneuvering and gimbal rotation,
    \item a reference frame for evaluating the target’s stability relative to the rotating scene,
    \item a qualitative separation between platform-induced dynamics and target-centric motion.
\end{itemize}
The fact that the target remains centered during substantial horizon rotation suggests that its 
apparent position is governed largely by the tracking system rather than by large-scale motion 
across the field of view.

\paragraph{Image Stability and Shape Consistency.}
The target’s infrared signature retains a stable morphology throughout the sequence. This stability 
enables limited geometric inference:
\begin{itemize}
    \item the absence of pronounced shape deformation argues against high-acceleration image-plane motion,
    \item the lack of abrupt centroid jumps suggests no strong high-frequency oscillatory behavior,
    \item coherent intensity structure indicates that residual motions are relatively smooth.
\end{itemize}
These observations provide qualitative evidence that whatever apparent motion exists is not 
dominated by jitter or tracking failures.

\paragraph{Relative Angular Behavior.}
By examining the orientation of the target relative to the rotating horizon, one can extract 
qualitative information about relative angular behavior:
\begin{itemize}
    \item whether the target exhibits slow drift relative to the horizon,
    \item whether its apparent alignment changes gradually or abruptly,
    \item whether any relative angular behavior is consistent across distinct phases of platform rotation.
\end{itemize}
These relative angular cues do not yield curvature or trajectory estimates but help distinguish 
smooth, coordinated evolution from irregular or noise-driven changes.

\paragraph{Limitations of Extractable Features.}
Crucially, the available geometric cues do \emph{not} permit:
\begin{itemize}
    \item computation of curvature $\kappa(t)$ or curvature rate $\dot{\kappa}(t)$,
    \item estimation of a center-of-curvature or radial profile,
    \item recovery of absolute velocity, acceleration, or jerk,
    \item construction of a consistent image-plane trajectory comparable to PR018.
\end{itemize}
For this reason, the GIMBAL analysis remains strictly qualitative. All inferences are framed in 
terms of relative stability, smoothness, and alignment, rather than quantitative kinematic 
measures.

\paragraph{Role Within the Unified Framework.}
Despite these limitations, the extractable features serve an important role: they allow us to 
evaluate whether the observable behavior of the target is qualitatively compatible with the kinds 
of smooth, persistent regimes associated with orb-like or turn-like motion in the unified 
framework, or whether it appears inconsistent with those regimes. These cues provide the basis for 
the qualitative state-consistency assessment in the next subsection.

\subsection{Qualitative State-Consistency Assessment}
\label{subsec:gimbal_state_consistency}

Because the GIMBAL dataset does not permit extraction of a reliable image-plane trajectory, the 
state-consistency analysis cannot rely on quantitative curvature $\kappa(t)$, curvature rate 
$\dot{\kappa}(t)$, or finite-difference kinematic derivatives. Instead, we evaluate whether the 
observable geometric cues identified in Section~\ref{subsec:gimbal_features} exhibit qualitative 
patterns consistent with any of the motion-state regimes defined in the unified framework.

This assessment is interpretive rather than computational, and is guided by three criteria:
\begin{enumerate}
    \item structural smoothness of the target’s apparent behavior,
    \item temporal persistence or coherence across distinct intervals of the recording,
    \item compatibility with the qualitative signatures of the Straight, Turn, Hover, and Orb states.
\end{enumerate}
The goal is not to assign a discrete state sequence, but to evaluate whether any observable 
behaviors contradict or align with the taxonomy developed in Part~II.

\paragraph{Straight-State Consistency.}
A motion would be \emph{qualitatively straight-like} if the target exhibited:
\begin{itemize}
    \item minimal displacement within the field of view,
    \item negligible relative drift against the rotating horizon,
    \item image stability with no indication of curvature-driven lateral motion.
\end{itemize}
GIMBAL does show periods of limited displacement, but these coincide with active sensor tracking 
and cannot be interpreted as evidence of straight-line intrinsic motion. No inconsistencies with a 
straight-state pattern are observed, but no affirmative geometric evidence supports it either.

\paragraph{Turn-State Consistency.}
A qualitative turn-like regime would require:
\begin{itemize}
    \item consistent lateral displacement relative to a background reference,
    \item monotonic angular change or persistent deviation from straight-like behavior,
    \item residual motion incompatible with pure stabilization.
\end{itemize}
Because horizon rotation is dominated by platform motion, the dataset does not present reliable 
turn-like cues. No geometric feature in GIMBAL contradicts a turn-like interpretation, but none 
provides positive support for one.

\paragraph{Hover-State Consistency.}
Hover-like behavior would include:
\begin{itemize}
    \item minimal intrinsic displacement,
    \item absence of systematic drift,
    \item frame-to-frame stability suggestive of low-speed or stationary dynamics.
\end{itemize}
The target’s near-centering and morphological stability could appear hover-like, but these are 
fully attributable to the sensor tracking system. Thus, hover-like patterns are \emph{not} ruled 
out, but cannot be meaningfully inferred.

\paragraph{Orb-State Consistency.}
The orb state is characterized by smoothly evolving curvature and persistence of dynamical regime 
beyond classical aerodynamic expectations. Qualitatively, this would manifest as:
\begin{itemize}
    \item smooth, coordinated evolution relative to a rotating reference frame,
    \item absence of ballistic straight-like segments,
    \item no abrupt kinematic transitions or discontinuities.
\end{itemize}
Although curvature cannot be computed for GIMBAL, two observable behaviors are noteworthy:
\begin{enumerate}
    \item the target’s apparent motion (to the extent it is visible) shows no abrupt jumps,
    \item its morphological stability persists during substantial platform-induced rotation.
\end{enumerate}
These observations are \emph{consistent} with smooth evolution, but they do not distinguish between 
hover-like, turn-like, or orb-like regimes. The dataset lacks the geometric resolution needed to 
affirm or reject orb-state behavior.

\paragraph{Synthesis.}
Across all four states, we find:
\begin{itemize}
    \item no observable behavior that contradicts any state class outright,
    \item insufficient geometric information to positively support any single state,
    \item qualitative compatibility with multiple regimes due to the dominance of parallax and tracking.
\end{itemize}
Thus, the appropriate conclusion is one of disciplined uncertainty:  
\textit{the GIMBAL dataset does not provide enough geometric information to determine which motion 
state, if any, the target occupies}.  
This stands in contrast to PR018, where curvature-based analysis enabled quantitative state 
assignment.

The qualitative consistency analysis therefore reinforces the conservative stance taken throughout 
this supplement: the GIMBAL sequence lacks the evidential content required for state classification 
but remains fully interpretable within the unified motion-state framework.

\subsection{Comparison With PR018 Results}
\label{subsec:gimbal_comparison}

A central purpose of this supplementary analysis is to clarify how the qualitative cues observable 
in the GIMBAL dataset relate to the quantitative curvature-based results obtained from PR018. 
Because the two datasets differ fundamentally in the availability of stabilization, parallax 
conditions, and metadata, any comparison must explicitly separate \emph{what is comparable} from 
\emph{what is not}.

\paragraph{Differences in Data Structure.}
PR018 provides:
\begin{itemize}
    \item a stable, target-centered field of view,
    \item sufficient residual motion to reconstruct a 2D image-plane trajectory,
    \item conditions enabling computation of curvature, curvature rate, and finite-difference 
          kinematic derivatives.
\end{itemize}
By contrast, GIMBAL provides:
\begin{itemize}
    \item strong parallax dominated by platform and gimbal motion,
    \item tracking behavior that suppresses visible target displacement,
    \item no reliable reference frame for trajectory reconstruction.
\end{itemize}
As a result, the unified framework can be \emph{applied quantitatively} to PR018, but only 
\emph{qualitatively} to GIMBAL.

\paragraph{Comparability of Motion-State Signatures.}
The motion-state analysis in PR018 was driven by curvature signatures, state-likelihood evaluation, 
and posterior temporal structure. None of these quantities can be computed for GIMBAL. Therefore:
\begin{itemize}
    \item no direct comparison of curvature $\kappa(t)$ or curvature continuity can be made,
    \item no state-probability sequence can be constructed for GIMBAL,
    \item no dynamical inferences (straight, turn, hover, orb) can be quantitatively aligned.
\end{itemize}
The two datasets are therefore \emph{not comparable at the level of dynamical-state inference}.

\paragraph{Comparability of Qualitative Behaviors.}
Even without a trajectory, certain qualitative behaviors can still be contrasted. We highlight three 
points of relevance:

\begin{enumerate}
    \item \textbf{Smoothness vs. discontinuity.}  
    PR018 exhibited smooth curvature evolution with no surviving straight-line segments after 
    stabilization. GIMBAL shows no abrupt centroid jumps or discontinuous behavior, but this 
    smoothness may arise solely from active tracking.

    \item \textbf{Persistence of geometric structure.}  
    In PR018, the orb-like regime persisted across hundreds of frames and was supported by 
    quantitative diagnostics. In GIMBAL, the target’s morphological stability is persistent but 
    does not reflect intrinsic motion.

    \item \textbf{Relative-frame behavior.}  
    PR018 provided a stable image-plane frame that reflected target motion. In GIMBAL, the 
    rotating horizon dominates the frame, preventing any equivalence in geometric interpretation.
\end{enumerate}

While some qualitative superficial similarities exist—such as stable centroid structure and absence 
of jitter—they do not imply dynamical similarity.

\paragraph{Constraints on Physical Interpretation.}
In PR018, the unified framework permitted elimination of several physical categories due to the 
observed curvature behavior (e.g., no ballistic segments). No such eliminations are possible for 
GIMBAL. Specifically:
\begin{itemize}
    \item GIMBAL cannot rule out straight, turn, hover, or orb-like regimes,
    \item GIMBAL cannot constrain acceleration or curvature profiles,
    \item GIMBAL cannot be used to evaluate dynamical feasibility of hypotheses.
\end{itemize}
This distinction is crucial for scientific rigor: the evidential strength of the two datasets is 
fundamentally unequal.

\paragraph{Synthesis and Interpretation.}
The comparison yields three disciplined conclusions:

\begin{enumerate}
    \item \textbf{PR018 and GIMBAL cannot be treated as equivalent for motion-state analysis.}  
    Only PR018 offers the necessary geometric observables for quantitative classification.

    \item \textbf{Qualitative cues in GIMBAL do not contradict the unified framework.}  
    GIMBAL shows no behavior incompatible with smooth or persistent regimes, but this carries no 
    dynamical weight.

    \item \textbf{GIMBAL provides a test of interpretive discipline rather than motion inference.}  
    Its value lies in confirming that the unified framework maintains methodological restraint in 
    the presence of incomplete data.
\end{enumerate}

Thus, while PR018 supports a quantitative demonstration of the unified motion-state model, the 
GIMBAL dataset serves as a complementary case illustrating how the framework operates under strong 
observational constraints without exceeding the evidential content of the data.

\subsection{Summary of GIMBAL Supplement}
\label{subsec:gimbal_summary}

The GIMBAL dataset provides a valuable but fundamentally limited test case for the unified 
motion-state framework. Unlike PR018, where approximate stabilization and sufficient residual 
motion enabled quantitative curvature analysis, the GIMBAL sequence is dominated by 
platform-induced parallax, active tracking behavior, and a lack of supporting telemetry. These 
conditions preclude reconstruction of an image-plane trajectory and prevent computation of 
curvature, curvature rate, or state-likelihood estimates.

Despite these constraints, several meaningful conclusions emerge from the qualitative analysis:

\paragraph{1. Geometric cues are available but non-quantitative.}
Features such as horizon rotation, target morphological stability, and small residual centroid 
motions provide limited geometric insight. However, these cues cannot be used to infer intrinsic 
target motion, and no dynamical quantities can be estimated.

\paragraph{2. No state classification is possible.}
Because curvature-based diagnostics cannot be computed, none of the four motion states 
(straight, turn, hover, orb) can be positively identified or excluded.  
The dataset is consistent with all regimes but affirmatively supports none.

\paragraph{3. Qualitative behaviors do not contradict the unified framework.}
The observable properties of the GIMBAL target—morphological stability, lack of jitter, smooth 
appearance relative to a rotating frame—are compatible with the conceptual structure of the 
framework. Crucially, this compatibility arises from insufficient data rather than evidence of 
smooth or persistent motion.

\paragraph{4. Comparison with PR018 highlights a contrast in evidential strength.}
PR018 provides direct geometric observables enabling quantitative state inference and elimination 
of incompatible physical models.  
GIMBAL does not.  
This contrast illustrates the importance of dataset structure in motion-state classification.

\paragraph{5. The supplement demonstrates interpretive discipline.}
A central contribution of the unified motion-state framework is its insistence on proportional 
inference: conclusions must scale with the quantity and reliability of available geometric data.  
The GIMBAL supplement shows that the framework remains fully applicable—but appropriately 
restrained—even when only partial information is accessible.

\paragraph{Overall Synthesis.}
The GIMBAL dataset does not support quantitative dynamical modeling or state assignment, but it 
plays an important complementary role in the manuscript:

\begin{itemize}
    \item it tests the unified framework’s ability to operate under severe observational constraints,
    \item it highlights the methodological distinction between geometric visibility and interpretive certainty,
    \item it reinforces the evidential strength of the PR018 analysis by contrast,
    \item it establishes clear scientific boundaries that prevent over interpretation.
\end{itemize}

In this sense, the supplementary analysis satisfies a critical epistemic function:  
\emph{it demonstrates that the unified motion-state framework is robust not only when data are 
rich but also when they are incomplete—while maintaining fidelity to the principles of 
quantitative geometry and conservative inference.}


