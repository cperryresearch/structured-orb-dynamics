\section{Application of Structured Orb Dynamics to Biological Migration Systems}
\subsection{VI-A.1 Scope and Intent}

This section explores whether the geometry-first, state-based framework of Structured Orb Dynamics (SOD) generalizes to biological migration systems when observed under partial and uncertain measurement conditions. The objective is not to interpret biological behavior, infer intent, or model navigational mechanisms, but rather to evaluate the descriptive coherence of the SOD state machinery when applied outside the context of UAP trajectory analysis.

Migration trajectories present a well-studied yet inference-constrained class of motion characterized by long-range displacement, intermittent observation, environmental perturbation, and variable sampling resolution. These properties make migration an appropriate testbed for assessing whether SOD’s kinematic state definitions and segmentation logic remain stable under known, non-anomalous conditions.

No claims regarding causation, optimization, evolutionary strategy, or biological decision-making are made in this section. The analysis is strictly descriptive and methodological in scope.

\subsection{VI-A.2 Migration as a Motion-Inference Problem}

From a kinematic perspective, biological migration can be treated as a motion-inference problem involving extended trajectories observed incompletely across time and space. Tracking data are often sparse, irregularly sampled, and subject to environmental interference, occlusion, or measurement uncertainty. Despite these limitations, migration datasets nonetheless exhibit structured movement patterns across multiple temporal and spatial scales.

This section adopts a geometry-first framing, treating migratory paths as sequences of observed positions without embedding domain-specific assumptions regarding motivation, navigation cues, or biological purpose. By abstracting migration in this way, the analysis remains agnostic to species-level differences and focuses solely on trajectory geometry and state transitions observable in the data.

This framing mirrors the constraints encountered in UAP footage analysis, albeit within a domain where the class of moving objects is known. Migration therefore serves as a controlled reference domain for evaluating the stability and generality of the SOD framework.

\subsection{VI-A.3 Mapping SOD Motion States to Migratory Trajectories}

The SOD framework defines motion in terms of discrete kinematic states inferred from trajectory geometry. These states may be mapped to migratory trajectories without introducing biological interpretation, as described below:

Straight: Sustained directional travel characterized by low curvature variance and consistent displacement over time.

Turn: Reorientation events marked by localized increases in curvature, corresponding to course corrections or directional changes.

Hover: Localized motion with minimal net displacement, consistent with stopover behavior or confined exploration.

Orb: A kinematic regime dominated by low velocity, limited displacement, or elevated observational uncertainty.

In this context, the Orb state denotes a descriptive regime defined by motion ambiguity or measurement limits rather than a behavioral or biological classification. Its inclusion preserves consistency with earlier sections while reinforcing the distinction between kinematic description and interpretive attribution.

\subsection{VI-A.4 Geometry-First Metrics and State Segmentation}

Trajectory segmentation within migration datasets may be performed using the same geometry-first metrics employed throughout SOD, including curvature-based time series, segment duration distributions, and state transition probabilities. These metrics are evaluated independently of species identity, environmental context, or assumed navigational strategy.

By applying identical segmentation logic across domains, SOD enables direct comparison between migration trajectories and other partially observed motion systems. This invariance is central to assessing whether the framework captures fundamental properties of motion geometry rather than domain-specific artifacts.

\subsection{VI-A.5 Illustrative Pipeline Using Migratory Bird GPS Data}

To illustrate the domain generality of the Structured Orb Dynamics (SOD) framework beyond UAP trajectory analysis, a conceptual application to biological migration is outlined using GPS-tracked migratory bird data from the Movebank repository. The datasets considered represent seasonal continental migration in passerine species, corresponding to small-to-medium migratory land birds observed over extended spatial and temporal scales.

In this pipeline, individual migration tracks are treated solely as sequences of observed positions under partial and uncertain sampling. No species-specific, ecological, or behavioral annotations are incorporated into the analysis. Trajectories are first projected into a common analysis space, after which geometry-based metrics are computed to characterize local curvature, displacement, and temporal structure.

State segmentation is then performed using the same kinematic criteria applied throughout the SOD framework, assigning segments to descriptive motion regimes (Straight, Turn, Hover, and Orb) based exclusively on trajectory geometry and uncertainty considerations. Resulting segmentations enable examination of state durations and transition structure without invoking biological intent or navigational mechanisms.

This conceptual example is presented to demonstrate methodological consistency and domain generality rather than to report empirical findings. Migration trajectories serve here as a known-class motion system against which the coherence and stability of the SOD segmentation machinery may be evaluated under realistic observation constraints.

\begin{figure}[t]
    \centering

    \begin{subfigure}[b]{0.32\textwidth}
        \centering
        \includegraphics[width=\textwidth]{figures/VI-A_Panel_A_Trajectory.png}
        \caption{Projected trajectory}
        \label{fig:vi-a-panel-a}
    \end{subfigure}
    \hfill
    \begin{subfigure}[b]{0.32\textwidth}
        \centering
        \includegraphics[width=\textwidth]{figures/VI-A_Panel_B_Curvature.png}
        \caption{Curvature proxy vs.\ time}
        \label{fig:vi-a-panel-b}
    \end{subfigure}
    \hfill
    \begin{subfigure}[b]{0.32\textwidth}
        \centering
        \includegraphics[width=\textwidth]{figures/VI-A_Panel_C_Segmented_OptionB.png}
        \caption{State-segmented trajectory}
        \label{fig:vi-a-panel-c}
    \end{subfigure}

    \caption{Geometry-first state segmentation of a partially observed trajectory.
    (A) Observed trajectory projected into analysis space without domain-specific assumptions regarding object class, intent, or environmental context.
    (B) Curvature time series derived from the projected trajectory, illustrating regions of sustained directional motion, localized reorientation, and intervals where sparse sampling increases uncertainty.
    (C) Trajectory segmented into Structured Orb Dynamics (SOD) kinematic states using geometry-based criteria. Low-displacement segments that are identifiable under sufficient temporal separation are classified as Hover and treated as a secondary sub-case within the segmentation logic. Segments dominated by sampling gaps or ambiguous geometry are classified as \emph{Orb}, reflecting uncertainty rather than inferred behavior.
    The trajectory shown corresponds to a GPS-tracked migratory passerine bird from publicly available Movebank data and is included as an illustrative example of a known-class motion system under partial observation.}
    \label{fig:vi-a-sod-migration}
\end{figure}

\subsection{VI-A.6 Relation to UAP Trajectory Analysis}

The inclusion of biological migration as an exploratory extension does not alter the interpretive constraints applied to UAP footage. Instead, migration trajectories provide a known-class reference for evaluating state segmentation behavior under partial observation.

While the same kinematic machinery may be applied across domains, the interpretive layer remains domain-specific. Migration serves as a validation context for assessing segmentation stability, whereas UAP footage represents an extreme uncertainty regime where object class and intent are not assumed.

\subsection{VI-A.7 Limitations and Non-Claims}

This section does not attempt to infer biological intent, model navigation mechanisms, evaluate energetic optimization, or explain migratory origin or purpose. The analysis remains confined to kinematic description under uncertainty.

Similarly, no claims are made regarding the explanatory power of SOD beyond motion characterization. The framework is not proposed as a theory of behavior or causation.

\subsection{VI-A.8 Implications for General Motion Analysis}

Demonstrating that SOD applies coherently to biological migration supports its characterization as a general motion-inference framework rather than a domain-specific tool. This extension broadens the potential applicability of SOD while preserving falsifiability, interpretive restraint, and methodological clarity.
