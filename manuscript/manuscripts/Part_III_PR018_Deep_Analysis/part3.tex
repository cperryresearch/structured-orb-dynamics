\section{PR018 Deep Analysis}
\label{sec:pr018_analysis}

This part applies the unified theoretical framework to the PR018 dataset, demonstrating how the 
classifier operates on real-world sensor data and evaluating the resulting dynamical-state 
interpretations.

\subsection{Dataset Overview and Reconstruction Summary}
\label{subsec:dataset_overview}

The PR018 dataset consists of a stabilized mid-wave infrared (MWIR) video released by the U.S. 
Department of Defense as part of the Unidentified Aerial Phenomena Task Force materials (2020). 
The video comprises approximately 2{,}781 frames and depicts a compact thermal signature moving 
across the field of view of an onboard sensor. No accompanying metadata—such as focal length, 
platform altitude, gimbal angles, or range estimates—was included in the public release.

Despite the absence of physical calibration parameters, the video exhibits several properties that 
enable geometric reconstruction of the object's apparent motion in the image plane:
\begin{itemize}
    \item the imagery is approximately stabilized around the target,
    \item frame timing appears consistent with a nominal $\sim 30$\,Hz sample rate,
    \item the target remains thermally saturated relative to the background,
    \item no major occlusions or dropouts occur within the analyzed interval.
\end{itemize}

\paragraph{Trajectory Extraction.}
The target's centroid was estimated on a frame-by-frame basis using intensity-weighted localization 
within a bounded search window. This produced a raw track in image-plane coordinates 
$\{x_{\text{raw}}(t)\}$ that represents the apparent two-dimensional motion of the source relative 
to the stabilized camera frame.

To reduce subpixel jitter and enhance geometric consistency, the raw track was smoothed using a 
low-order temporal filter. The resulting processed trajectory $x(t)$ serves as the basis for all 
subsequent kinematic and curvature-based analysis.

\paragraph{Derivative and Feature Estimation.}
Velocity, acceleration, curvature, and curvature rate were computed from $x(t)$ using the 
finite-difference and smoothing procedures described in Part~I and formalized in 
Section~\ref{subsec:observation_model}. These operations yielded the geometric feature sequence
\[
    f_t = \big(\kappa(t),\, \dot{\kappa}(t),\, \|v(t)\|,\, \text{radial}(t),\, a(t),\, j(t) \big),
\]
which forms the input to the unified likelihood model in Part~II.

\paragraph{Uncertainties and Assumptions.}
Because physical metadata such as range and altitude are unavailable, all geometric quantities are 
interpreted in *relative* rather than absolute units. This does not affect the classifier, which 
operates entirely on dimensionless geometric features. However, it does imply:
\begin{itemize}
    \item no claims about physical speed or distance can be made,
    \item only *shape* and *structure* of motion are analyzed,
    \item dynamical interpretations pertain to image-plane geometry.
\end{itemize}

The analysis therefore focuses on reproducible geometric signatures rather than on physical 
modeling or energetic estimation.

\paragraph{Purpose of the Dataset in This Framework.}
The PR018 dataset functions as a representative example of a stabilized infrared tracking scenario 
with incomplete metadata. It provides a realistic test case for the unified classifier, allowing us 
to evaluate whether the theoretical model of curvature, state transitions, and likelihood inference 
yields coherent dynamical descriptions under actual sensor conditions.

\subsection{Geometric Feature Extraction}
\label{subsec:feature_extraction}

The unified motion-state framework operates on geometric features derived from the processed 
trajectory $x(t)$. These quantities describe the local structure of motion in the image plane and 
serve as the inputs to the likelihood and state-transition model formalized in Part~II. This 
section summarizes the extraction of curvature, curvature rate, velocity, acceleration, and radial 
indicators from the PR018 data.

\paragraph{Velocity and Acceleration.}
Frame-to-frame velocities were computed using smoothed finite differences of the trajectory:
\[
v(t) = \frac{x(t+\Delta t) - x(t)}{\Delta t}.
\]
Second differences yield the acceleration estimate $a(t)$. Smoothing was applied both before and 
after derivative computation to mitigate amplification of sensor noise and tracking jitter.

\paragraph{Curvature.}
Curvature $\kappa(t)$ provides a scale-invariant measure of instantaneous trajectory bending:
\[
\kappa(t) = \frac{|x'(t) y''(t) - y'(t) x''(t)|}
                 {\big(x'(t)^2 + y'(t)^2 \big)^{3/2}}.
\]

\paragraph{Curvature Rate.}
The curvature derivative $\dot{\kappa}(t)$ highlights dynamical transitions not evident from 
curvature alone.

\paragraph{Radial Motion Indicators.}
We evaluate whether the trajectory exhibits bounded motion around a local center via
\[
r(t) = \| x(t) - c(t) \|,
\]
where $c(t)$ is a locally estimated curvature center.

\paragraph{Summary of Extracted Features.}
The resulting feature sequence
\[
f_t = \left(\kappa(t), \dot{\kappa}(t), \|v(t)\|, a(t), r(t)\right)
\]
forms a geometrically consistent description of the motion.

\subsection{State Classification Results}
\label{subsec:state_results}

Using the geometric feature sequence from 
Section~\ref{subsec:feature_extraction}, the unified likelihood model produced per-frame posterior 
probabilities:
\[
p_t = \big( p_t(S),\, p_t(T),\, p_t(H),\, p_t(O) \big).
\]

\paragraph{Dominant Orb Classification.}
The orb state $O$ receives the largest posterior support across the sequence.

\paragraph{Secondary Turn Segments.}
Turn-state probabilities $p_t(T)$ increase in intervals of temporarily stabilized curvature.

\paragraph{Limited Straight or Hover Support.}
Straight and hover states receive negligible posterior support.

\paragraph{Posterior Structure.}
The posterior landscape shows coherent, stable state identification consistent with the geometric 
features.

\subsection{Posterior Probability Interpretation}
\label{subsec:posterior_interpretation}

High-confidence orb intervals persist throughout most of the sequence. Transitional intervals 
capture ambiguity rather than classification instability. The posterior landscape remains stable 
across smoothing parameters.

\subsection{Consistency With Unified Framework}
\label{subsec:framework_consistency}

The PR018 results align strongly with all theoretical expectations of the unified framework.

\subsection{Summary of PR018 Analysis}
\label{subsec:pr018_summary}

The PR018 dataset demonstrates that the unified classifier provides a stable, interpretable, and 
geometrically grounded description of the observed motion. Smooth curvature evolution, bounded 
radial variation, and consistent higher-order structure strongly favor the orb regime. Transitional 
intervals between orb and turn states correspond to meaningful changes in curvature evolution.

These results serve two primary roles:
\begin{itemize}
    \item establishing a concrete demonstration of the unified framework under realistic sensor 
          conditions, and
    \item providing a reference pattern for comparison to other datasets such as GIMBAL.
\end{itemize}

The PR018 and GIMBAL analyses together outline the operational boundaries of the unified 
motion-state framework, demonstrating both its strengths and its appropriate limits.
