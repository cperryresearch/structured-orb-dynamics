% ==============================
% PART II – THEORETICAL FRAMEWORK
% ==============================

Part~II develops the mathematical and conceptual foundations of the Orb Motion Classifier.
Where Part~I focused on constructing empirical trajectories, estimating curvature, and evaluating
state likelihoods, this part formalizes the geometric principles that make those operations
coherent. Here we introduce the theoretical structure that explains why the classifier behaves
as it does, how curvature and its derivatives encode dynamical information, and under what
conditions the resulting motion-state framework can generalize across heterogeneous datasets
with incomplete metadata.

The goal of this part is to unify curvature-based diagnostics, discrete motion-state modeling, and 
probabilistic classification under a single coherent analytical framework. This enables rigorous 
interpretation of state transitions, diagnostic consistency across datasets, and transparent 
assumptions for future extensions.

% --------------------------------
% UNIFIED THEORY FRAMEWORK SECTION
% --------------------------------

\section{Unified Theory Framework}
\label{sec:unified_framework}

This section introduces the theoretical architecture that unifies curvature-based motion analysis, 
discrete dynamical state modeling, and probabilistic classification into a single interpretable 
framework. The objective is to establish a mathematically consistent foundation for analyzing 
unknown aerial trajectories and comparing their dynamical signatures across heterogeneous datasets.

% --------------------------------
% 2.1 Overview of Motion-State Geometry
% --------------------------------

\subsection{Overview of Motion-State Geometry}
\label{subsec:geometry_overview}

We define four fundamental motion states characterized by geometric invariants:

\begin{itemize}
    \item \textbf{Straight State ($S$)}: low curvature, stable direction of motion.
    \item \textbf{Turn State ($T$)}: sustained nonzero curvature with consistent turning direction.
    \item \textbf{Hover State ($H$)}: minimal displacement and low-speed regime.
    \item \textbf{Orb State ($O$)}: smoothly varying curvature with continuous higher derivatives.
\end{itemize}

Each state corresponds to a qualitatively distinct dynamical behavior and is identified through 
features derived from velocity, acceleration, and curvature signals.

These geometric distinctions form the foundation of the unified framework by mapping observable 
trajectory features to interpretable dynamical categories.

% --------------------------------
% 2.2 Curvature as a State Variable
% --------------------------------

\subsection{Curvature as a State Variable}
\label{subsec:curvature_state}

Curvature plays a central role in the unified motion-state framework because it provides a 
coordinate-invariant measure of how a trajectory departs from straight-line motion. Unlike 
velocity or acceleration—both of which depend on external reference frames—curvature is an 
intrinsic geometric property of the path itself. It therefore offers a stable discriminator of 
motion regimes, even when sensor metadata or absolute position information is incomplete.

Let $x(t) = (x_1(t), x_2(t))$ denote the reconstructed 2D trajectory in image-plane coordinates. 
The instantaneous curvature is defined as:
\begin{equation}
    \kappa(t) = 
    \frac{|x'(t) y''(t) - y'(t) x''(t)|}
    {\big( x'(t)^2 + y'(t)^2 \big)^{3/2}},
    \label{eq:curvature}
\end{equation}
with the standard extension to 3D trajectories as needed. Curvature evaluates how sharply the 
trajectory bends at each moment; $\kappa(t) \approx 0$ corresponds to straight-line motion, while 
larger values indicate progressively tighter turning.

\paragraph{Curvature Rate and Jerk Minimization.}
The temporal derivative $\dot{\kappa}(t)$ captures how curvature evolves over time. In many 
natural and engineered systems, high-frequency fluctuations in $\dot{\kappa}(t)$ are suppressed 
due to energetic or physical constraints, producing trajectories that minimize jerk. This yields:
\begin{itemize}
    \item slowly varying curvature in straight and turning motion,
    \item noisy or unstable curvature estimates during hover,
    \item smooth, bounded curvature evolution in the orb regime.
\end{itemize}

\paragraph{Curvature Under Measurement Noise.}
Because curvature depends on first and second derivatives, it is sensitive to noise. To mitigate 
this, curvature estimation proceeds through:
\begin{enumerate}
    \item temporal smoothing of the trajectory,
    \item finite-difference derivative estimation,
    \item normalization to account for varying speed,
    \item thresholding when velocity approaches zero.
\end{enumerate}

These steps preserve the qualitative structure of the curvature trace while suppressing 
frame-to-frame irregularities. Even under moderate noise, curvature retains the essential 
patterns needed to discriminate between straight, turning, hovering, and orb-like motion.

\paragraph{Why Curvature is Foundational.}
Curvature is the logical basis for the unified model because it offers:
\begin{itemize}
    \item \textbf{Geometric Invariance:} independence from coordinate frames.
    \item \textbf{State Separation:} each motion regime produces a distinct curvature profile.
    \item \textbf{Probabilistic Compatibility:} $\kappa(t)$ and $\dot{\kappa}(t)$ integrate 
          naturally into likelihood models.
\end{itemize}

Curvature therefore provides an interpretable, noise-tolerant bridge between observable kinematic 
features and the dynamical state model described next.

% --------------------------------
% 2.3 Discrete Dynamical State Model
% --------------------------------

\subsection{Discrete Dynamical State Model}
\label{subsec:state_model}

While curvature provides an instantaneous geometric description of motion, many behaviors of 
interest unfold over extended time intervals. To capture these temporal dependencies, we model 
the evolving motion regime as a discrete state process
\[
    S_t \in \mathcal{S} = \{ S,\, T,\, H,\, O \},
\]
where $S$, $T$, $H$, and $O$ denote the straight, turn, hover, and orb states, respectively. The 
state variable $S_t$ summarizes the dominant dynamical behavior at time $t$, integrating local 
geometric cues with temporal smoothing that reflects physical continuity.

\paragraph{State-Transition Structure.}
We assume that state evolution follows a first-order Markov process:
\[
    \mathbb{P}(S_{t} = s' \mid S_{t-1} = s,\, x_{1:t}) 
    = \mathbb{P}(S_{t} = s' \mid S_{t-1} = s),
\]
where $x_{1:t}$ denotes all observations up to time $t$. Although the classifier incorporates 
observational likelihoods, the transition structure itself depends only on the preceding state. 
This assumption reflects the notion that sudden, high-frequency transitions between strongly 
different dynamical regimes are physically improbable.

The transition matrix encodes:
\begin{itemize}
    \item high self-transition probabilities (temporal persistence),
    \item moderate transitions among compatible states (e.g., $S \leftrightarrow T$),
    \item suppressed transitions between incompatible states (e.g., $H \rightarrow S$ without 
          acceleration),
    \item rare transitions into or out of the orb state, reflecting its specialized geometry.
\end{itemize}

\paragraph{State-Conditional Likelihood Models.}
Each state generates characteristic patterns in curvature, velocity, and acceleration. We model 
the likelihood of observing the geometric features $f_t$ at time $t$ under each state as
\[
    \mathcal{L}(f_t \mid S_t = s),
\]
with $f_t$ drawn from the quantities estimated in Part I, including curvature $\kappa(t)$, 
curvature rate $\dot{\kappa}(t)$, speed $\|v(t)\|$, radial boundedness indicators, and derivative 
signatures.

Informally:
\begin{itemize}
    \item Straight motion ($S$) favors $\kappa(t) \approx 0$ with small $\dot{\kappa}(t)$.
    \item Turning motion ($T$) favors sustained nonzero curvature with sign consistency.
    \item Hovering ($H$) yields low speeds and unstable curvature estimates.
    \item Orb motion ($O$) favors smooth, bounded curvature with coherent evolution over time.
\end{itemize}

These likelihood models do not enforce deterministic boundaries; instead, they assign weights 
reflecting how well each state explains the observed geometry.

\paragraph{Physical and Dynamical Constraints.}
The state model incorporates minimal structural assumptions motivated by physical feasibility:
\begin{itemize}
    \item Hover ($H$) cannot transition abruptly to high-speed turning without intermediate 
          acceleration.
    \item Orb motion ($O$) requires multiple consecutive frames of bounded radial displacement.
    \item Straight and turn states ($S$ and $T$) may interleave but typically preserve short-term 
          curvature trends.
    \item State changes must occur over time scales longer than the sampling interval, unless 
          curvature or velocity exhibit clear discontinuities.
\end{itemize}

These constraints improve robustness by suppressing implausible sequences caused by noise or 
momentary reconstruction artifacts.

\paragraph{Dynamical Interpretation.}
The discrete state process provides an interpretable temporal summary of the trajectory. Rather 
than treating each frame independently, the Markov model links adjacent estimates, promoting 
temporal coherence and enabling identification of multi-frame regimes such as:
\begin{itemize}
    \item extended straight-line traversals,
    \item arcs or turning maneuvers,
    \item hovering intervals,
    \item sustained orbital patterns.
\end{itemize}

This structure forms the backbone of the unified classification framework. By encoding motion 
behavior as a sequence of discrete, interpretable states, the model creates a bridge between raw 
kinematic measurements and the higher-level dynamical diagnostics developed throughout Part II.

% --------------------------------
% 2.4 Observation Model
% --------------------------------

\subsection{Observation Model}
\label{subsec:observation_model}

The unified framework relies on geometric quantities—such as curvature, curvature rate, velocity, 
and radial displacement—that are derived from visual observations rather than direct physical 
measurements. As a result, the accuracy and interpretability of the model depend critically on the 
observation pipeline that reconstructs the trajectory $x(t)$ from the underlying sensor data. This 
section formalizes the assumptions and procedures that govern this reconstruction.

\paragraph{Platform-Motion Compensation.}
Raw sensor imagery often reflects the combined motion of the target and the recording platform. 
To isolate the target's apparent motion in the image plane, we apply stabilization procedures that 
compensate for platform drift, gimbal rotation, and camera jitter. Although the PR-018 dataset lacks 
complete metadata, frame-to-frame alignment via optical or structural features provides an 
approximate but effective means of isolating relative motion. The resulting stabilized track serves 
as the input for curvature and derivative estimation.

\paragraph{Noise Characterization.}
Visual tracking is inherently noisy due to factors including:
\begin{itemize}
    \item optical-flow estimation error,
    \item thermal noise and sensor quantization,
    \item compression artifacts,
    \item partial occlusions or frame cropping.
\end{itemize}

We treat these noise sources as zero-mean disturbances that perturb the observed position. Their 
primary effect is to introduce short-scale variability in velocity and second derivatives. Because 
curvature depends on these derivatives, subsequent smoothing is essential to recovering meaningful 
geometric structure.

\paragraph{Finite-Difference Derivative Estimation.}
Velocity and acceleration are estimated using centered finite differences over the stabilized 
trajectory:
\[
    v(t) \approx \frac{x(t+\Delta t) - x(t-\Delta t)}{2\Delta t}, \qquad
    a(t) \approx \frac{x(t+\Delta t) - 2x(t) + x(t-\Delta t)}{\Delta t^2}.
\]
These estimates define the curvature $\kappa(t)$ and curvature rate $\dot{\kappa}(t)$ introduced 
in Sections~\ref{subsec:geometry_overview}–\ref{subsec:curvature_state}. Finite differences are 
sensitive to noise but provide unbiased estimates under mild smoothness assumptions. Their 
simplicity also ensures reproducibility and transparency across datasets.

\paragraph{Temporal Smoothing of Geometric Quantities.}
To mitigate derivative amplification of noise, we apply temporal smoothing to $x(t)$ or to the 
derived velocity and curvature sequences. Smoothing may be implemented using moving averages, 
Savitzky–Golay filtering, or low-order polynomial fits, depending on dataset resolution. The 
objective is to preserve low-frequency geometric structure—such as sustained curvature or gradual 
turning—while suppressing frame-level jitter.

\paragraph{Mapping Observations to State Features.}
The observation model defines the sequence of feature vectors
\[
    f_t = \big(\kappa(t),\, \dot{\kappa}(t),\, \|v(t)\|,\, \text{radial}(t),\, a(t),\, j(t) \big),
\]
where “radial” denotes boundedness indicators and $j(t)$ the estimated jerk. These features serve as 
inputs to the likelihood models of Section~\ref{subsec:state_model}. Importantly, the observation model 
does not assume perfect accuracy: instead, it is designed to generate stable geometric summaries 
that remain robust under moderate noise.

\paragraph{Interpretive Role.}
By formalizing how raw sensor data is transformed into geometric descriptors, the observation model 
provides a transparent foundation for the unified framework. It clarifies the assumptions underlying 
trajectory reconstruction, identifies sources of uncertainty, and ensures that curvature-based 
classification remains grounded in observable quantities rather than unverified metadata or 
assumptions about sensor configuration.

% --------------------------------
% 2.5 Orb-State Justification
% --------------------------------

\subsection{Orb-State Justification}
\label{subsec:orb_justification}

The introduction of the orb state is motivated by empirical patterns in curvature and radial 
structure that cannot be adequately explained by classical categories such as straight motion, 
turning, or hovering. The orb state captures a regime in which the trajectory exhibits smooth, 
continuous curvature evolution together with bounded radial displacement relative to a local 
center. This combination of properties produces a distinctive dynamical signature that persists 
across multiple frames.

\paragraph{Curvature Evolution.}
Unlike standard turning motion—which is characterized by approximately constant curvature over a 
segment—the orb regime features curvature that varies continuously while remaining bounded away 
from zero. This produces a smooth “wavelike” evolution of $\kappa(t)$ and $\dot{\kappa}(t)$, 
reflecting gradual reorientation rather than rigid circular turning or erratic curvature noise. 
Empirically, such curvature traces display:
\begin{itemize}
    \item sustained nonzero curvature,
    \item coherent temporal evolution with low jerk,
    \item inflection patterns inconsistent with constant-radius turns,
    \item resilience to frame-level noise even after smoothing.
\end{itemize}

These features cannot be reliably modeled as either straight or turning motion.

\paragraph{Radial Boundedness.}
A defining characteristic of orbital-like motion is approximate radial boundedness: the trajectory
remains within a finite neighborhood of a time-varying local center. This condition is weaker than 
true circular motion—no strict radius or periodicity is assumed—but it distinguishes the orb state 
from turning, which typically lacks consistent radial structure. Radial boundedness is estimated 
using:
\begin{itemize}
    \item local center-of-curvature approximations,
    \item short-horizon estimates of displacement from inferred centers,
    \item variance thresholds on radial deviation.
\end{itemize}

When curvature is smooth and the radius of curvature fluctuates within bounded limits, the 
trajectory exhibits a recognizable orbital pattern.

\paragraph{Distinction From Hovering and Noise-Dominated Motion.}
Hovering motion often produces unstable curvature estimates because small positional noise creates 
large derivative fluctuations. In contrast, the orb state maintains stable curvature even at slow 
speeds due to coherent geometric structure. The orb state therefore cannot be explained as a 
noise artifact or jitter amplification and instead reflects meaningful kinematic organization.

\paragraph{Physical Non-Commitment.}
The orb state is not intended to imply any specific propulsion mechanism, control system, or 
underlying physics. It is a descriptive motion category defined solely by geometric and temporal 
signatures. The framework does not infer intent, internal structure, or energetic constraints; it 
merely classifies observable trajectory features into a consistent state taxonomy.

\paragraph{Diagnostic Value.}
Introducing the orb state improves classification accuracy and interpretability by preventing 
mis-labeling of coherent, bounded-curvature regimes as either turning or hovering. This yields:
\begin{itemize}
    \item cleaner temporal segmentation,
    \item reduced state ambiguity,
    \item improved likelihood separation,
    \item more informative downstream inference.
\end{itemize}

The orb state therefore plays a central role in the unified framework by capturing a distinct,
empirically motivated mode of motion that emerges naturally from curvature-based analysis.

% --------------------------------
% 2.6 Unified Likelihood Model
% --------------------------------

\subsection{Unified Likelihood Model}
\label{subsec:likelihood_model}

The unified likelihood model integrates geometric features, temporal dependencies, and state 
priors into a coherent probabilistic framework for motion-state classification. Given the feature 
sequence $\{f_t\}$ extracted from the observation model, the goal is to infer the posterior 
probabilities
\[
    \mathbb{P}(S_t = s \mid f_{1:t})
\]
for each state $s \in \mathcal{S}$. This section formalizes the likelihood structure that links 
geometric observations to the discrete state process.

\paragraph{Feature Likelihoods.}
For each state $s$, we define a state-conditional likelihood
\[
    \mathcal{L}(f_t \mid S_t = s),
\]
which evaluates how well the observed geometric features at time $t$ agree with the characteristic 
patterns of state $s$. These features include:
\begin{itemize}
    \item curvature $\kappa(t)$,
    \item curvature rate $\dot{\kappa}(t)$,
    \item speed $\|v(t)\|$,
    \item acceleration magnitude $\|a(t)\|$,
    \item radial boundedness indicators,
    \item jerk estimates $j(t)$.
\end{itemize}

The likelihood models are intentionally simple—typically Gaussian or log-normal components—
ensuring interpretability and robustness across heterogeneous datasets.

\paragraph{State Priors and Transition Dynamics.}
The Markovian structure introduced in Section~\ref{subsec:state_model} contributes a prior 
distribution on state evolution:
\[
    \mathbb{P}(S_t = s' \mid S_{t-1} = s) = T_{s,s'},
\]
where $T$ is the transition matrix. The transition priors encode temporal smoothness by 
favoring:
\begin{itemize}
    \item self-transitions (persistence of motion regimes),
    \item transitions between compatible states (e.g., straight-to-turn),
    \item rare transitions into or out of the orb state without geometric justification.
\end{itemize}

These priors regularize the classification process, reducing sensitivity to noise and preventing 
implausible state-switching behavior.

\paragraph{Joint Likelihood and Posterior Inference.}
Combining likelihoods and priors yields the joint probability of the state sequence and feature 
sequence:
\[
    \mathbb{P}(S_{1:T}, f_{1:T}) 
    = \mathbb{P}(S_1) \prod_{t=2}^T T_{S_{t-1},S_t}
      \prod_{t=1}^T \mathcal{L}(f_t \mid S_t).
\]

Posterior inference proceeds by computing:
\[
    \mathbb{P}(S_t \mid f_{1:T}),
\]
using standard dynamic programming techniques such as the forward–backward algorithm. This ensures 
that every state assignment reflects both local geometric evidence and global temporal coherence.

\paragraph{Interpretability of Posterior Probabilities.}
Posterior state probabilities provide an intuitive and transparent summary of the classifier's 
output. They allow researchers to:
\begin{itemize}
    \item identify dominant motion regimes over time,
    \item visualize confidence in state transitions,
    \item localize ambiguous or noisy segments,
    \item compare dynamical behavior across datasets.
\end{itemize}

Rather than producing a single hard label, the classifier outputs a probability distribution over 
states at each frame, allowing uncertainty to be represented explicitly.

\paragraph{Unified Structure.}
The unified likelihood model links together the major components of the theoretical framework:
\begin{itemize}
    \item curvature-based geometry (Sections~\ref{subsec:geometry_overview}--\ref{subsec:curvature_state}),
    \item temporal state dynamics (Section~\ref{subsec:state_model}),
    \item observation and noise modeling (Section~\ref{subsec:observation_model}),
    \item empirical justification for the orb regime (Section~\ref{subsec:orb_justification}).
\end{itemize}

This integration yields a physically grounded, statistically coherent model capable of describing 
unknown aerial trajectories with interpretable structure and quantifiable uncertainty.

% --------------------------------
% 2.7 Generalization Across Datasets
% --------------------------------

\subsection{Generalization Across Datasets}
\label{subsec:generalization}

A key objective of the unified framework is to provide consistent, interpretable motion-state 
classification across heterogeneous datasets. Although different sensors yield trajectories with 
varying resolutions, noise characteristics, and metadata completeness, the core geometric structure 
of the model remains invariant. This section outlines the elements of the framework that generalize 
across datasets and the adjustments required for practical application.

\paragraph{Geometric Invariance of Curvature-Based Features.}
Curvature, curvature rate, and radial structure are intrinsic properties of the trajectory and do 
not depend on absolute position, sensor calibration, or platform-specific metadata. As a result:
\begin{itemize}
    \item curvature-based motion signatures are comparable across sensors,
    \item feature extraction remains stable even when range or altitude are unknown,
    \item local geometric patterns provide a common basis for classification.
\end{itemize}

This invariance enables coherent interpretation of motion states despite differences in imaging 
modalities or acquisition conditions.

\paragraph{Robustness to Missing Metadata.}
Many publicly released sensor products lack complete information about focal length, gimbal 
calibration, frame timing, or platform motion. The framework accommodates such gaps by relying on:
\begin{itemize}
    \item relative motion rather than absolute physical units,
    \item smoothing and derivative estimation that operate purely on image-plane coordinates,
    \item likelihood models built on dimensionless geometric features.
\end{itemize}

This ensures that the classifier remains functional and interpretable even when only stabilized 
frame sequences are available, as in the PR-018 and GIMBAL datasets.

\paragraph{Dataset-Specific Noise Profiles.}
Different sensors introduce distinct noise characteristics—thermal noise in infrared imagery, 
compression artifacts in digital video, or tracking jitter in parallax-limited sequences. The 
observation model accommodates such variations through:
\begin{itemize}
    \item dataset-specific smoothing parameters,
    \item adaptive thresholds for curvature stability,
    \item variance-based weighting of derivative estimates.
\end{itemize}

Because the underlying geometric features are stable, only minimal adjustments to smoothing or 
noise filtering are required for each dataset.

\paragraph{State Interpretation Consistency.}
The state definitions introduced in Section~\ref{subsec:geometry_overview} remain valid across 
datasets. Straight, turn, hover, and orb states correspond to universal geometric regimes of 
trajectory evolution, independent of sensor modality or environmental conditions. As a result:
\begin{itemize}
    \item state sequences can be compared across datasets,
    \item posterior probabilities reflect consistent dynamical signatures,
    \item transitions carry the same interpretive meaning regardless of sensor source.
\end{itemize}

This consistency is especially important for evaluating whether different datasets exhibit 
qualitatively similar motion behavior.

\paragraph{Application to Multi-Sensor Cases.}
In sequences such as GIMBAL, parallax, platform rotation, and partial stabilization introduce 
additional geometric distortions. While these distortions affect the absolute shape of the 
trajectory, the unified framework remains applicable because:
\begin{itemize}
    \item the model uses local geometric features rather than global trajectory shape,
    \item curvature and curvature rate remain informative after smoothing,
    \item bounded-curvature regimes (orb states) manifest even under partial stabilization.
\end{itemize}

Dataset-specific corrections—such as pre-filtering or parallax compensation—can refine the 
trajectory but are not required for the core classifier to operate.

\paragraph{Summary.}
The unified framework generalizes across datasets because it is grounded in geometric features 
that are invariant under changes in sensor configuration, platform motion, and metadata 
availability. This flexibility supports consistent dynamical interpretation across PR-018, GIMBAL, 
and future datasets with similar characteristics.

% --------------------------------
% 2.8 Implications for Motion Interpretation
% --------------------------------

\subsection{Implications for Motion Interpretation}
\label{subsec:implications}

The unified framework provides a principled foundation for interpreting unknown aerial trajectories 
strictly in terms of observable geometric and dynamical structure. By grounding the analysis in 
curvature, state transitions, and probabilistic inference, the classifier avoids assumptions about 
propulsion, intent, or underlying mechanisms. This section summarizes the interpretive implications 
of the model and clarifies what conclusions may—and may not—be drawn from state sequences.

\paragraph{Constraints on Kinematic Hypotheses.}
Each motion state imposes specific geometric and dynamical constraints on feasible trajectories:
\begin{itemize}
    \item Straight segments ($S$) indicate persistent directional motion with minimal reorientation.
    \item Turning segments ($T$) reflect coordinated changes in direction with sustained curvature.
    \item Hover intervals ($H$) imply low-speed, small-displacement regimes.
    \item Orb segments ($O$) denote smooth, bounded-curvature evolution consistent with motion 
          around a local center.
\end{itemize}

These constraints allow researchers to evaluate which classes of kinematic models—ballistic, 
aerodynamic, noise-driven, or otherwise—are compatible with the observed state sequence.

\paragraph{Avoiding Overinterpretation.}
The classifier is not designed to infer causality or mechanism. For example:
\begin{itemize}
    \item an orb segment does not imply controlled flight or intent,
    \item a straight segment does not imply ballistic motion,
    \item a hovering interval does not imply levitation.
\end{itemize}

Instead, the classifier describes the *form* of motion, not its *origin*. Interpretation beyond 
kinematic structure requires additional physical information not present in image-plane trajectories.

\paragraph{State Sequences as Dynamical Summaries.}
State sequences serve as compact representations of the trajectory, enabling:
\begin{itemize}
    \item identification of dominant motion regimes,
    \item segmentation of complex trajectories into interpretable units,
    \item comparison of dynamical patterns across datasets,
    \item localization of anomalous or transitional intervals.
\end{itemize}

Posterior probabilities further highlight periods of ambiguity or uncertainty, allowing careful 
evaluation of borderline or noise-dominated segments.

\paragraph{Comparative Analysis Across Datasets.}
Because the states are defined geometrically, their interpretation remains consistent when applied 
to different sensor products. A sequence classified as $O \rightarrow T \rightarrow S$ carries the 
same dynamical meaning in PR-018 as in GIMBAL, even if the underlying sensor geometries differ. 
This supports cross-dataset comparison and strengthens the generality of the framework.

\paragraph{Limitations and Scope.}
The unified model is intentionally conservative. It operates entirely within the observational 
limits of image-plane trajectories and does not rely on assumptions about range, altitude, 
propulsion, or mass. As such:
\begin{itemize}
    \item physical inference beyond geometry is out of scope,
    \item energetic modeling is not attempted,
    \item mechanistic interpretation is avoided.
\end{itemize}

The framework is therefore most valuable as a diagnostic and comparative tool, not as a complete 
physical model of aerial motion.

\paragraph{Summary.}
The implications of the unified framework lie in its ability to provide repeatable, interpretable, 
and dataset-agnostic descriptions of motion behavior. By separating geometric observation from 
physical speculation, the model establishes a rigorous foundation for analyzing unknown aerial 
trajectories and preparing them for further scientific study.
